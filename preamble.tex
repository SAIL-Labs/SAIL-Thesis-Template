% ----------------------------------------------------------------------
%                   LATEX TEMPLATE FOR THESIS
% ----------------------------------------------------------------------

% based on Harish Bhanderi's PhD/MPhil template, then Uni Cambridge
% http://www-h.eng.cam.ac.uk/help/tpl/textprocessing/ThesisStyle/
% corrected and extended in 2007 by Jakob Suckale, then MPI-CBG PhD programme
% and made available through OpenWetWare.org - the free biology wiki


%: Style file for Latex
% Most style definitions are in the external file PhDthesisPSnPDF.
% In this template package, it can be found in ./Latex/Classes/

% DRAFT BOX
\ifthesisisdraft
	\renewcommand{\submittedtext}{{A thesis {\bf draft} for the degree of}}
	\fancyhead[RO]{\framebox{{\sc Draft:}\quad\datethis}\hfill \bfseries\rightmark}
	\fancyhead[LE]{\framebox{{\sc Draft:}\quad\datethis}\hfill \bfseries\leftmark}
\fi
% Additional Packages
\usepackage[T1]{fontenc}
\usepackage{lipsum}
\usepackage[disable]{todonotes} %disable
%\usepackage[marginpar]{todo}
\usepackage{Latex/Macros/aas_macros}
\usepackage[numbered,autolinebreaks]{Latex/StyleFiles/mcode}
\usepackage[palatino]{quotchap}
\usepackage[section,below]{placeins}
\usepackage{listings}
% Load ctable after tikz (tikz loaded by todonotes or other packages)
\usepackage{ctable}
\usepackage{datetime}
\newdateformat{monthyeardate}{%
  \monthname[\THEMONTH], \THEYEAR}
\usepackage{ragged2e}
\usepackage{siunitx}

\usepackage{tablefootnote}



\newcounter{mycomment}
\newcommand{\mycomment}[1]{%
    % initials of the author (optional) + note in the margin
    \refstepcounter{mycomment}%
    {%
        \todo[color={blue!20!white},size=\scriptsize]{%
		%\begin{spacing}{0.5}
            \textbf{[\uppercase{CHB}\themycomment]:}~#1} 
		%\end{spacing}%
	}
}

\newcounter{SLScomment}
\newcommand{\SLScomment}[1]{%
    % initials of the author (optional) + note in the margin
    \refstepcounter{mycomment}%
    {%
        \todo[color={red!20!white},size=\scriptsize]{%
		%\begin{spacing}{0.5}
            \textbf{[\uppercase{SLS}\themycomment]:}~#1} 
		%\end{spacing}%
	}
}

\newcounter{JGRcomment}
\newcommand{\JGRcomment}[1]{%
    % initials of the author (optional) + note in the margin
    \refstepcounter{mycomment}%
    {%
        \todo[color={green!20!white},size=\scriptsize]{%
		%\begin{spacing}{0.5}
            \textbf{[\uppercase{JGR}\themycomment]:}~#1} 
		%\end{spacing}%
	}
}

%: ----------------------------------------------------------------------
%:                  Colours
% -----------------------------------------------------------------------
\definecolor{SchoolColor}{RGB}{230, 70, 38} % Masterbrand red
\definecolor{chaptergrey}{RGB}{230, 70, 38}
\definecolor{midgrey}{RGB}{66, 66, 66}

\definecolor{mygreen}{rgb}{0,0.6,0}
\definecolor{mygray}{rgb}{0.5,0.5,0.5}
\definecolor{mymauve}{rgb}{0.58,0,0.82}

%: ----------------------------------------------------------------------
%:                  Listings (code inculision)
% -----------------------------------------------------------------------

\lstset{ %
  backgroundcolor=\color{white},   % choose the background color; you must add \usepackage{color} or \usepackage{xcolor}
  basicstyle=\normalsize\ttfamily,%\tiny,        % the size of the fonts that are used for the code
  breakatwhitespace=false,         % sets if automatic breaks should only happen at whitespace
  breaklines=true,                 % sets automatic line breaking
  postbreak=\raisebox{0ex}[0ex][0ex]{\ensuremath{\color{red}\hookrightarrow\space}}, % ornage arrow to contiunue line
  captionpos=t,                    % sets the caption-position to bottom
  commentstyle=\color{mygreen},    % comment style
  deletekeywords={...},            % if you want to delete keywords from the given language
  escapeinside={\%*}{*)},          % if you want to add LaTeX within your code
  extendedchars=true,              % lets you use non-ASCII characters; for 8-bits encodings only, does not work with UTF-8
  frame=none,                      % adds a frame around the code
  keepspaces=true,                 % keeps spaces in text, useful for keeping indentation of code (possibly needs columns=flexible)
  keywordstyle=\color{blue},       % keyword style
  language=Matlab,                 % the language of the code
  morekeywords={*,classdef,misprint, self, handleAllHidden, properties, methods,bsxfun,repmat},            % if you want to add more keywords to the set
  numbers=left,                    % where to put the line-numbers; possible values are (none, left, right)
  numbersep=4pt,                   % how far the line-numbers are from the code
  numberstyle=\tiny\color{mygray}, % the style that is used for the line-numbers
  numberfirstline=true,
  rulecolor=\color{gray},         % if not set, the frame-color may be changed on line-breaks within not-black text (e.g. comments (green here))
  showspaces=false,                % show spaces everywhere adding particular underscores; it overrides 'showstringspaces'
  showstringspaces=false,          % underline spaces within strings only
  showtabs=false,                  % show tabs within strings adding particular underscores
  stepnumber=1,                    % the step between two line-numbers. If it's 1, each line will be numbered
  stringstyle=\color{mymauve},     % string literal style
  tabsize=2,                       % sets default tabsize to 2 spaces
  title=\lstname,                  % show the filename of files included with \lstinputlisting; also try caption instead of title
%  xleftmargin=2cm,
%  xrightmargin=2cm,
%  framexleftmargin=2cm,
%  framexrightmargin=2cm,
%  framexbottommargin=4pt,
}
\renewcommand\lstlistingname{Code}
\renewcommand\lstlistlistingname{Code}
 
\lstnewenvironment{codebox}[1][] 
   	{\centering\lstset{linewidth=\textwidth,
					xleftmargin=0.075\textwidth,
				   xrightmargin=0.075\textwidth,
						  frame=single, 
				     basicstyle=\footnotesize,
				     abovecaptionskip=-0.75cm,
				       #1}
	}{}

%: Macro file for Latex
% Macros help you summarise frequently repeated Latex commands.
% Here, they are placed in an external file /Latex/Macros/MacroFile1.tex
% An macro that you may use frequently is the figuremacro (see introduction.tex)
% This file contains macros that can be called up from connected TeX files
% It helps to summarise repeated code, e.g. figure insertion (see below).

% words

% insert a centered figure with caption and description
% parameters 1:filename, 2:title, 3:description and label

%% referencing macros, enforces standard labelling practice.
\newcommand{\fig}[1]{Fig.~\ref{fig:#1}}
\newcommand{\Fig}[1]{Figure~\ref{fig:#1}}
\newcommand{\eqn}[1]{Eqn.~\ref{eqn:#1}}
\newcommand{\Eqn}[1]{Equation~\ref{eqn:#1}}
\newcommand{\sect}[1]{Sec.~\ref{sect:#1}}
\newcommand{\chap}[1]{Chap.~\ref{chap:#1}}
\newcommand{\tabl}[1]{Table~\ref{tabl:#1}}
\newcommand{\apdx}[1]{Appendix.~\ref{apdx:#1}}


%% quick formating word commands
\newcommand{\echelle}{\'{e}chelle\xspace}
\newcommand{\alfcen}{$\alpha$-\text{Cen}\xspace}

\newcommand{\wrt}{with respect to\xspace}
\newcommand{\fcam}{$f_\text{cam}$\xspace}
\newcommand{\fcamMath}{f_\text{cam}}
\newcommand{\ewidth}{$1/e^2$~diameter\xspace}

%units shortcuts
\newcommand{\lpmm}[1]{\SI{#1}{lines\per\milli\meter}\xspace}
\newcommand{\mum}[1]{~$#1\mu$m}
\newcommand{\um}{~$\mu$m\xspace}
\newcommand{\nanometre}{~nm\xspace}
\newcommand{\millimetre}{~mm\xspace}
\newcommand{\picometre}{~pm\xspace}

%% punctuation fixs
\renewcommand{\~}{\raise.17ex\hbox{$\scriptstyle\sim$}}
\usepackage{xspace}
\newcommand*{\eg}{e.g.\@\xspace}
\newcommand*{\ie}{i.e.\@\xspace}


%% usefull



\makeatletter
\newcommand*{\etc}{%
    \@ifnextchar{.}%
        {etc}%
        {etc.\@\xspace}%
}
\makeatother

\newcommand{\todofigure}[1]{\missingfigure{#1}\todo{#1}}


\usepackage{pdfpages}
\newcommand{\insertpaper}[3]%
           { % Syntax: \insertpaper{title}{label}{File}
             % Requires: tocloft hyperref pdfpages
            \includepdf[addtotoc={1,section,1,#1,#2}]{#3}
            \includepdf[pages=2-last]{#3}
           }

%Inkscape inclusions
\usepackage{bashful}
\usepackage{import}
%\usepackage{ifplatform}
\usepackage{etoolbox}

\newbool{canInkscape} %default false
\setbool{canInkscape}{true}

\newcommand{\includesvg}[1]{
    \ifbool{canInkscape}{
        \immediate\write18{sh tools/inkscapeSVG2PDFtex.sh #1}
        }{}%
\IfFileExists{#1.pdf_tex}{\textsf{\input{#1.pdf_tex}}}{\textsf{#1.pdf\_tex is missing. inkscapeSVG2PDFtex.sh likely failed.  \\ \rule{5cm}{5cm}}}
}


%fig macros
\newcommand{\figuremacro}[3]{
	\begin{figure}[b]
		\centering
		\includegraphics[width=1\textwidth]{#1}
		\caption[#2]{{#2} - #3}
		\label{#1}
	\end{figure}
}

% insert a centered figure with caption and description AND WIDTH
% parameters 1:filename, 2:title, 3:description and label, 4: textwidth
% textwidth 1 means as text, 0.5 means half the width of the text
\newcommand{\figuremacroW}[4]{
	\begin{figure}
		\centering
		\includegraphics[width=#4\textwidth]{#1}
		\caption[#2]{#3}
		\label{fig:#1}
	\end{figure}
}

\newcommand{\figuremacroWp}[5]{
	\begin{figure}[#5]
		\centering
		\includegraphics[width=#4\textwidth]{#1}
		\caption[#2]{#3}
		\label{fig:#1}
	\end{figure}
}

% insert a centered subfigure with caption and description AND WIDTH
% parameters 1:filename, 2:title, 3:description and label, 4: textwidth
% textwidth 1 means as text, 0.5 means half the width of the text
%\newcommand{\figuremacroSUB}[7]{
%	\begin{figure}[tbh]
%	   \centering
%		\subfloat[]{\label{#2}\includegraphics[width=#4\textwidth]{#2}}
%		\quad
%		\subfloat[]{\label{#5}\includegraphics[width=#7\textwidth]{#5}}
%	   \caption[#1]{{#1} - {\it a)} #3  {\it b)} #6}
%	   \label{#1}
%	\end{figure}
%}

%1:title 2:file1 3:caption1 4:width1 5:file2 6:caption2 7:width2
\newcommand{\figuremacroSUB}[7]{
\begin{figure}[tbh]
\centering
     % subfig 1
	 \begin{subfigure}[b]{#4\textwidth}
	     \centering
         \includegraphics[width=\textwidth]{#2}
         \caption{}
         \label{fig:#2}
	 \end{subfigure} 
	 % subfig 2
	 \begin{subfigure}[b]{#7\textwidth}
	     \centering
         \includegraphics[width=\textwidth]{#5}
         \caption{}
         \label{fig:#5}
	 \end{subfigure}
	 	 
	 \vspace{-0.2cm}
	 \caption[#1]{{\it a)} #3  {\it b)} #6}
	 \label{fig:#1}
\end{figure}
}

\newcommand{\figuremacroSUBp}[8]{
\begin{figure}[#8]
\centering
     % subfig 1
	 \begin{subfigure}[b]{#4\textwidth}
	     \centering
         \includegraphics[width=\textwidth]{#2}
         \caption{}
         \label{fig:#2}
	 \end{subfigure} 
	 % subfig 2
	 \begin{subfigure}[b]{#7\textwidth}
	     \centering
         \includegraphics[width=\textwidth]{#5}
         \caption{}
         \label{fig:#5}
	 \end{subfigure}
	 
	 \vspace{-0.2cm}
	 \caption[#1]{{\it a)} #3  {\it b)} #6}
	 \label{fig:#1}
\end{figure}
}


%\figuremacroSUB{title}{file1}{caption1}{size1}{file2}{caption2}{size2}

% inserts a figure with wrapped around text; only suitable for NARROW figs
% o is for outside on a double paged document; others: l, r, i(inside)
% text and figure will each be half of the document width
% note: long captions often crash with adjacent content; take care
% in general: above 2 macro produce more reliable layout
\newcommand{\figuremacroN}[3]{
	\begin{wrapfigure}{o}{0.5\textwidth}
		\centering
		\includegraphics[width=0.48\textwidth]{#1}
		\caption[#2]{{\small\textbf{#2} - #3}}
		\label{#1}
	\end{wrapfigure}
}

% predefined commands by Harish
\newcommand{\PdfPsText}[2]{
  \ifpdf
     #1
  \else
     #2
  \fi
}

\newcommand{\IncludeGraphicsH}[3]{
  \PdfPsText{\includegraphics[height=#2]{#1}}{\includegraphics[bb = #3, height=#2]{#1}}
}

\newcommand{\IncludeGraphicsW}[3]{
  \PdfPsText{\includegraphics[width=#2]{#1}}{\includegraphics[bb = #3, width=#2]{#1}}
}

\newcommand{\InsertFig}[3]{
  \begin{figure}[!htbp]
    \begin{center}
      \leavevmode
      #1
      \caption{#2}
      \label{#3}
    \end{center}
  \end{figure}
}

%% fix biblatex's fullcite to use maxbibnames
%\makeatletter
%\DeclareCiteCommand{\fullcite}
%  {\defcounter{maxnames}{\blx@maxbibnames}%
%    \usebibmacro{prenote}}
%  {\usedriver
%     {\DeclareNameAlias{sortname}{default}}
%     {\thefield{entrytype}}}
%  {\multicitedelim}
%  {\usebibmacro{postnote}}
%\DeclareCiteCommand{\footfullcite}[\mkbibfootnote]
%  {\defcounter{maxnames}{\blx@maxbibnames}%
%    \usebibmacro{prenote}}
%  {\usedriver
%     {\DeclareNameAlias{sortname}{default}}
%     {\thefield{entrytype}}}
%  {\multicitedelim}
%  {\usebibmacro{postnote}}
%\makeatother


%%% Local Variables: 
%%% mode: latex
%%% TeX-master: "~/Documents/LaTeX/CUEDThesisPSnPDF/thesis"
%%% End: 



%: ----------------------------------------------------------------------
%:                  TITLE PAGE: name, degree,..
% -----------------------------------------------------------------------
% below is to generate the title page with crest and author name

% Document metadata - define once, use everywhere
% \def\thesistitle{A new approach to classical spectroscopy.}
% \def\thesisauthor{Christopher H. Betters}
% \def\thesisauthoremail{chris@chrisbetters.com}
% \def\thesiskeywords{Astrophotonics; Diffraction-limited spectroscopy; PhD Thesis}
% \def\thesislogo{0_frontmatter/figures/usydlogo_bw.pdf}
% \def\thesiscollege{Faculty of Science}
% \def\thesisuniversity{The University of Sydney}
% \def\thesisdegree{Doctor of Philosophy}


% Document setup using dynamic variables
\title{Your amazing thesis title.}

% ----------------------------------------------------------------------
% The section below defines www links/email for author and institutions
% They will appear on the title page of the PDF and can be clicked
\author{First Lastname}
\def\thesisauthoremail{First.Lastname@sydney.edu.au}
\def\thesiskeywords{Science; PhD Thesis}


\collegeordept{Faculty of Science}
\university{The University of Sydney}
\crest{
  \includegraphics[width=6cm]{0_frontmatter/figures/usydlogo_bw.pdf}
}

\degree{Doctor of Philosophy}
\degreedate{\monthyeardate\today}
\degreeyear{\the\year}
% ----------------------------------------------------------------------

%PDF output configuration using dynamic variables
\hypersetup{pdfinfo={
Title={\thetitle},
Author={\theauthor\space\thesisauthoremail},
Keywords={\thesiskeywords}
},
colorlinks,
citecolor=SchoolColor,
filecolor=black,
linkcolor=black,
urlcolor=SchoolColor}

       
% turn of those nasty overfull and underfull hboxes
\hbadness=10000
\hfuzz=50pt

%: --------------------------------------------------------------
%:                         REFERENCES
%: --------------------------------------------------------------
%\addbibresource{bib/references.bib} % must have .bib
%tex
%\addbibresource{bib/allofpapers.bib}
%\addbibresource{bib/referencesSPIE2012-1.bib}
%\addbibresource{bib/referencesSPIE2012-2.bib}

\addbibresource{bib/thesisRefs.bib}
\addbibresource{bib/fake_publications.bib}

\makeglossaries