% this file is called up by thesis.tex
% content in this file will be fed into the main document

%\newacronym{<label>}{<abbrv>}{<full>}


%\newglossaryentry{naiive}
%{
%  name=na\"{\i}ve,
%  description={is a French loanword (adjective, form of naïf)
%               indicating having or showing a lack of experience,
%               understanding or sophistication},
%  sort=naive
%}


% todo Glossary Shortcut
%% locaization stuuf

\newif\ifamerican

\americanfalse

\ifamerican
  \newcommand{\fibre}{fiber\xspace}
  \newcommand{\fibres}{fibers\xspace}
\else
  \newcommand{\fibre}{fibre\xspace}
  \newcommand{\fibres}{fibres\xspace}
\fi


%\newacronym{smf}{SMF}{single-mode \fibre\xspace}
\newglossaryentry{smf}
{
  name={SMF},
  description={single-mode \fibre},
  first={single-mode \fibre (\glsentrytext{smf})},
  plural={SMFs},
  descriptionplural={single-mode \fibres},
  firstplural={single-mode \fibres (\glsentryplural{smf})}
}
\newcommand{\smf}{\gls{smf}\xspace}
\newcommand{\smfs}{\glspl{smf}\xspace}


\newacronym{sm}{SM}{single-mode\xspace}
\newcommand{\sm}{\gls{sm}\xspace}

\newacronym{mm}{MM}{multi-mode\xspace}
\newcommand{\MM}{\gls{mm}\xspace}

%\newacronym{mmf}{MMF}{multi-mode \fibre}
\newglossaryentry{mmf}
{
  name={MMF},
  description={multi-mode \fibre},
  first={multi-mode \fibre (\glsentrytext{mmf})},
  plural={MMFs},
  descriptionplural={multi-mode \fibres},
  firstplural={multi-mode \fibres (\glsentryplural{mmf})}
}
\newcommand{\MMF}{\gls{mmf}\xspace}
\newcommand{\MMFs}{\glspl{mmf}\xspace}

\newacronym{pimms}{PIMMS}{photonic integrated multi-mode spectrograph}
\newcommand{\pimms}{\gls{pimms}\xspace}

\newacronym{mcf}{MCF}{multi-core fibre}
\newcommand{\mcf}{\gls{mcf}\xspace}

\newacronym{fwhm}{FWHM}{full width half maximum}
\newcommand{\fwhm}{\gls{fwhm}\xspace}

\newacronym{ifu}{IFU}{integral field unit}
\newcommand{\ifu}{\gls{ifu}\xspace}

\newacronym{uli}{ULI}{ultrafast laser inscription}
\newcommand{\uli}{\gls{uli}\xspace}

\newacronym{vph}{VPH}{volume-phase holographic}
\newcommand{\vph}{\gls{vph}\xspace}

\newacronym{ft}{FT}{Fourier transform}
\newcommand{\ft}{\gls{ft}\xspace}

\newacronym{ar}{AR}{anti-reflective}
\newcommand{\AR}{\gls{ar}\xspace}

\newacronym{fts}{FTS}{Fourier transform spectrometer}
\newcommand{\fts}{\gls{fts}\xspace}
\newcommand{\ftss}{\glspl{fts}\xspace}

\newacronym{mos}{MOS}{multi-object spectrograph}
\newcommand{\mos}{\gls{mos}\xspace}
\newcommand{\moss}{\glspl{mos}\xspace}

\newacronym{psf}{PSF}{point spread function}
\newcommand{\psf}{\gls{psf}\xspace}
\newcommand{\psfs}{\glspl{psf}\xspace}

\newacronym{lsf}{LSF}{line spread function}
\newcommand{\lsf}{\gls{psf}\xspace}

\newacronym{snr}{SNR}{single-to-noise ratio}
\newcommand{\snr}{\gls{snr}\xspace}

\newacronym{rms}{RMS}{root mean square}
\newcommand{\rms}{\gls{rms}\xspace}

\newacronym{fsr}{FSR}{free spectral range}
\newcommand{\fsr}{\gls{fsr}\xspace}

\newacronym{na}{NA}{numerical aperture}
\newcommand{\NA}{\gls{na}\xspace}

\newacronym{cots}{COTS}{commercial off-the-shelf}
\newcommand{\COTS}{\gls{cots}\xspace}

\newacronym{pcb}{PCB}{printed circuit board}
\newcommand{\pcb}{\gls{pcb}\xspace}
\newcommand{\pcbs}{\glspl{pcb}\xspace}

\newacronym{pop}{POP}{physical optics propagation}
\newcommand{\pop}{\gls{pop}\xspace}

\newacronym{pcf}{PCF}{photonic crystal fibre}
\newcommand{\pcf}{\gls{pcf}\xspace}

\newacronym{efl}{EFL}{effective focal length}
\newcommand{\efl}{\gls{efl}\xspace}
\newcommand{\efls}{\glspl{efl}\xspace}

\newacronym{thar}{ThAr}{Thorium-Argon}
\newcommand{\thar}{\gls{thar}\xspace}

\newacronym{fp}{FP}{Fabry-P\'erot}
\newcommand{\FabryPerot}{\gls{fp}\xspace}

\newacronym{ao}{AO}{adaptive optics}
\newcommand{\AO}{\gls{ao}\xspace}

\newacronym{tec}{TEC}{thermoelectric cooler}
\newcommand{\tec}{\gls{tec}\xspace}

\newacronym{ptv}{PTV}{peak to valley}
\newcommand{\ptv}{\gls{ptv}\xspace}

\newacronym{dft}{DFT}{direct Fourier transfrom}
\newcommand{\DFT}{\gls{dft}\xspace}

\newacronym{fft}{FFT}{fast Fourier transfrom}
\newcommand{\fft}{\gls{fft}\xspace}

\newacronym{adu}{ADU}{analogue-to-digital units}
\newcommand{\ADU}{\gls{adu}\xspace}

\newglossaryentry{pl}{
    name={PL},
    description={photonic lantern; a multi-mode to single-mode converter},
    first={photonic lantern (PL)}
}
\newcommand{\lantern}{\gls{pl}\xspace}
\newcommand{\lanterns}{\glspl{pl}\xspace}
\newcommand{\pl}{\gls{pl}\xspace}
\newcommand{\pls}{\glspl{pl}\xspace}

\newglossaryentry{awg}{
    name={AWG},
    description={arrayed waveguide grating; a device with a series single mode waveguides where each waveguide has increases in length by a fixed phase offset, producing a high order diffraction/dispersion},
    first={arrayed waveguide grating (AWG)}
}
\newcommand{\awg}{\gls{awg}\xspace}
\newcommand{\awgs}{\glspl{awg}\xspace}


\newglossaryentry{ips}{
    name={IPS},
    description={integrated photonic spectrograph; spectrograph composed of integrated photonics technologies, i.e. \uli photonics lantern fused to \awg},
    first={integrated photonic spectrograph (IPS)}
}
\newcommand{\ips}{\gls{ips}\xspace}

\newglossaryentry{leo}{
    name={LEO},
    description={low earth orbit is considered to be between \SIrange{160}{2000}{\kilo\meter}},
    first={low earth orbit (LEO)}
}
\newcommand{\leo}{\gls{leo}\xspace}


\newglossaryentry{pimms0}{
    name={PIMMS\#0},
    description={Hybrid (photonic and bulk optic) version of PIMMS},
}
\newcommand{\pimmsZ}{\gls{pimms0}\xspace}

\newglossaryentry{pimms1}{
    name={PIMMS\#1},
    description={Fully photonic version of PIMMS},
}
\newcommand{\pimmsOne}{\gls{pimms1}\xspace}

\newglossaryentry{etendue}{
    name={\'{e}tendue},
    symbol={$A\Omega$},
    description={geometric characterisation of an optical systems ability to accept light. Defined as product of area and solid angle of acceptance}
    }
\newcommand{\etendue}{\gls{etendue}\xspace}
\newcommand{\AOm}{\glssymbol{etendue}\xspace}

\newglossaryentry{throughput}{
    name={throughput},
    description={ratio of photons detected to photons collected.},
    first={throughput}
}
\newcommand{\throughput}{\gls{throughput}\xspace}

\newglossaryentry{resolvingpower}{
    name={resolving power},
    symbol={{$\mathcal{R}$}},
    description={quantitative measure of a spectrographs ability to resolve neighbouring spectral features. Defined as $\mathcal{R}=\lambda/\Delta\lambda$},
    first={resolving power ($\mathcal{R}$)}
}
\newcommand{\R}{\glssymbol{resolvingpower}}
\newcommand{\resolvingpower}{\gls{resolvingpower}\xspace}
\newcommand{\Rmath}{\mathcal{R}}
\newcommand{\Rdef}{$\lambda/\Delta\lambda$}


\newglossaryentry{spectralresolution}{
    name={spectral resolution},
    symbol={$\Delta\lambda$},
    description={measure of the smallest wavelength difference resolvable at wavelength $\lambda$. Typically measured as the \fwhm of a spectral feature}
}
\newcommand{\dlam}{\glssymbol{spectralresolution}\xspace}
\newcommand{\spectralresolution}{\gls{spectralresolution}\xspace}


\newglossaryentry{f-ratio}{
    name={focal ratio},
    symbol={$f/\#$},
    description={ratio of focal length and aperture/pupil diameter, defined as $f/\# = F/D$}
}
\newcommand{\fnum}{\glssymbol{f-ratio}\xspace}
\newcommand{\fratio}{\gls{f-ratio}\xspace}

\newglossaryentry{seeing}{
    name={seeing},
    description={the blurring of a telescope image due to atmospheric turbulence and varying the optical refractive index}
}
\newcommand{\seeing}{\gls{seeing}\xspace}

\newglossaryentry{swir}{
	name={SWIR},
    first={short-wave infrared (SWIR)},
    description={light with wavelengths from 1\um to 3\um}
}
\newcommand{\swir}{\gls{swir}\xspace}

\newglossaryentry{ir}{
	name={IR},
    first={infrared (IR)},
    description={light with wavelengths from 0.7\um to 100\um, but often used in context to indicate smaller regions.}
}
\newcommand{\ir}{\gls{ir}\xspace}

\newglossaryentry{nir}{
	name={NIR},
    first={near-infrared (NIR)},
    description={light with wavelength 0.7\um to 1\um, or edge of human vision to the silicon sensitivity cutoff}
}
\newcommand{\nir}{\gls{nir}\xspace}

\newglossaryentry{uv}{
	name={UV},
    first={ultraviolet (UV)},
    description={light with wavelength 10~nm to 400~nm.}
}
\newcommand{\uv}{\gls{uv}\xspace}


\newglossaryentry{mfd}{
    name={MFD},
    description={diameter of the intensity profile of light at the out put of \fibre. Normally measured at the width at the $1/e^2$ intensity level},
    first={mode-field diameter (MFD)}
}
\newcommand{\mfd}{\gls{mfd}\xspace}

%\newglossaryentry{}{
%    name={},
%    symbol={$$},
%    description={},
%	 first={near-infrared (NIR)}
%}
%\newcommand{}{\glssymbol{}\xspace}




