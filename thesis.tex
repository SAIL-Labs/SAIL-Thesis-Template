\documentclass[oneside,11pt,a4paper]{Latex/Classes/PhDthesisPSnPDF}
%\thesisisdrafttrue
\thesisisdraftfalse
% ----------------------------------------------------------------------
%                   LATEX TEMPLATE FOR THESIS
% ----------------------------------------------------------------------

% based on Harish Bhanderi's PhD/MPhil template, then Uni Cambridge
% http://www-h.eng.cam.ac.uk/help/tpl/textprocessing/ThesisStyle/
% corrected and extended in 2007 by Jakob Suckale, then MPI-CBG PhD programme
% and made available through OpenWetWare.org - the free biology wiki


%: Style file for Latex
% Most style definitions are in the external file PhDthesisPSnPDF.
% In this template package, it can be found in ./Latex/Classes/

% DRAFT BOX
\ifthesisisdraft
	\renewcommand{\submittedtext}{{A thesis {\bf draft} for the degree of}}
	\fancyhead[RO]{\framebox{{\sc Draft:}\quad\datethis}\hfill \bfseries\rightmark}
	\fancyhead[LE]{\framebox{{\sc Draft:}\quad\datethis}\hfill \bfseries\leftmark}
\fi
% Additional Packages
\usepackage[T1]{fontenc}
\usepackage{lipsum}
\usepackage[disable]{todonotes} %disable
%\usepackage[marginpar]{todo}
\usepackage{Latex/Macros/aas_macros}
\usepackage[numbered,autolinebreaks]{Latex/StyleFiles/mcode}
\usepackage[palatino]{quotchap}
\usepackage[section,below]{placeins}
\usepackage{listings}
% Load ctable after tikz (tikz loaded by todonotes or other packages)
\usepackage{ctable}
\usepackage{datetime}
\newdateformat{monthyeardate}{%
  \monthname[\THEMONTH], \THEYEAR}
\usepackage{ragged2e}
\usepackage{siunitx}

\usepackage{tablefootnote}



\newcounter{mycomment}
\newcommand{\mycomment}[1]{%
    % initials of the author (optional) + note in the margin
    \refstepcounter{mycomment}%
    {%
        \todo[color={blue!20!white},size=\scriptsize]{%
		%\begin{spacing}{0.5}
            \textbf{[\uppercase{CHB}\themycomment]:}~#1} 
		%\end{spacing}%
	}
}

\newcounter{SLScomment}
\newcommand{\SLScomment}[1]{%
    % initials of the author (optional) + note in the margin
    \refstepcounter{mycomment}%
    {%
        \todo[color={red!20!white},size=\scriptsize]{%
		%\begin{spacing}{0.5}
            \textbf{[\uppercase{SLS}\themycomment]:}~#1} 
		%\end{spacing}%
	}
}

\newcounter{JGRcomment}
\newcommand{\JGRcomment}[1]{%
    % initials of the author (optional) + note in the margin
    \refstepcounter{mycomment}%
    {%
        \todo[color={green!20!white},size=\scriptsize]{%
		%\begin{spacing}{0.5}
            \textbf{[\uppercase{JGR}\themycomment]:}~#1} 
		%\end{spacing}%
	}
}

%: ----------------------------------------------------------------------
%:                  Colours
% -----------------------------------------------------------------------
\definecolor{SchoolColor}{RGB}{230, 70, 38} % Masterbrand red
\definecolor{chaptergrey}{RGB}{230, 70, 38}
\definecolor{midgrey}{RGB}{66, 66, 66}

\definecolor{mygreen}{rgb}{0,0.6,0}
\definecolor{mygray}{rgb}{0.5,0.5,0.5}
\definecolor{mymauve}{rgb}{0.58,0,0.82}

%: ----------------------------------------------------------------------
%:                  Listings (code inculision)
% -----------------------------------------------------------------------

\lstset{ %
  backgroundcolor=\color{white},   % choose the background color; you must add \usepackage{color} or \usepackage{xcolor}
  basicstyle=\normalsize\ttfamily,%\tiny,        % the size of the fonts that are used for the code
  breakatwhitespace=false,         % sets if automatic breaks should only happen at whitespace
  breaklines=true,                 % sets automatic line breaking
  postbreak=\raisebox{0ex}[0ex][0ex]{\ensuremath{\color{red}\hookrightarrow\space}}, % ornage arrow to contiunue line
  captionpos=t,                    % sets the caption-position to bottom
  commentstyle=\color{mygreen},    % comment style
  deletekeywords={...},            % if you want to delete keywords from the given language
  escapeinside={\%*}{*)},          % if you want to add LaTeX within your code
  extendedchars=true,              % lets you use non-ASCII characters; for 8-bits encodings only, does not work with UTF-8
  frame=none,                      % adds a frame around the code
  keepspaces=true,                 % keeps spaces in text, useful for keeping indentation of code (possibly needs columns=flexible)
  keywordstyle=\color{blue},       % keyword style
  language=Matlab,                 % the language of the code
  morekeywords={*,classdef,misprint, self, handleAllHidden, properties, methods,bsxfun,repmat},            % if you want to add more keywords to the set
  numbers=left,                    % where to put the line-numbers; possible values are (none, left, right)
  numbersep=4pt,                   % how far the line-numbers are from the code
  numberstyle=\tiny\color{mygray}, % the style that is used for the line-numbers
  numberfirstline=true,
  rulecolor=\color{gray},         % if not set, the frame-color may be changed on line-breaks within not-black text (e.g. comments (green here))
  showspaces=false,                % show spaces everywhere adding particular underscores; it overrides 'showstringspaces'
  showstringspaces=false,          % underline spaces within strings only
  showtabs=false,                  % show tabs within strings adding particular underscores
  stepnumber=1,                    % the step between two line-numbers. If it's 1, each line will be numbered
  stringstyle=\color{mymauve},     % string literal style
  tabsize=2,                       % sets default tabsize to 2 spaces
  title=\lstname,                  % show the filename of files included with \lstinputlisting; also try caption instead of title
%  xleftmargin=2cm,
%  xrightmargin=2cm,
%  framexleftmargin=2cm,
%  framexrightmargin=2cm,
%  framexbottommargin=4pt,
}
\renewcommand\lstlistingname{Code}
\renewcommand\lstlistlistingname{Code}
 
\lstnewenvironment{codebox}[1][] 
   	{\centering\lstset{linewidth=\textwidth,
					xleftmargin=0.075\textwidth,
				   xrightmargin=0.075\textwidth,
						  frame=single, 
				     basicstyle=\footnotesize,
				     abovecaptionskip=-0.75cm,
				       #1}
	}{}

%: Macro file for Latex
% Macros help you summarise frequently repeated Latex commands.
% Here, they are placed in an external file /Latex/Macros/MacroFile1.tex
% An macro that you may use frequently is the figuremacro (see introduction.tex)
% This file contains macros that can be called up from connected TeX files
% It helps to summarise repeated code, e.g. figure insertion (see below).

% words

% insert a centered figure with caption and description
% parameters 1:filename, 2:title, 3:description and label

%% referencing macros, enforces standard labelling practice.
\newcommand{\fig}[1]{Fig.~\ref{fig:#1}}
\newcommand{\Fig}[1]{Figure~\ref{fig:#1}}
\newcommand{\eqn}[1]{Eqn.~\ref{eqn:#1}}
\newcommand{\Eqn}[1]{Equation~\ref{eqn:#1}}
\newcommand{\sect}[1]{Sec.~\ref{sect:#1}}
\newcommand{\chap}[1]{Chap.~\ref{chap:#1}}
\newcommand{\tabl}[1]{Table~\ref{tabl:#1}}
\newcommand{\apdx}[1]{Appendix.~\ref{apdx:#1}}


%% quick formating word commands
\newcommand{\echelle}{\'{e}chelle\xspace}
\newcommand{\alfcen}{$\alpha$-\text{Cen}\xspace}

\newcommand{\wrt}{with respect to\xspace}
\newcommand{\fcam}{$f_\text{cam}$\xspace}
\newcommand{\fcamMath}{f_\text{cam}}
\newcommand{\ewidth}{$1/e^2$~diameter\xspace}

%units shortcuts
\newcommand{\lpmm}[1]{\SI{#1}{lines\per\milli\meter}\xspace}
\newcommand{\mum}[1]{~$#1\mu$m}
\newcommand{\um}{~$\mu$m\xspace}
\newcommand{\nanometre}{~nm\xspace}
\newcommand{\millimetre}{~mm\xspace}
\newcommand{\picometre}{~pm\xspace}

%% punctuation fixs
\renewcommand{\~}{\raise.17ex\hbox{$\scriptstyle\sim$}}
\usepackage{xspace}
\newcommand*{\eg}{e.g.\@\xspace}
\newcommand*{\ie}{i.e.\@\xspace}


%% usefull



\makeatletter
\newcommand*{\etc}{%
    \@ifnextchar{.}%
        {etc}%
        {etc.\@\xspace}%
}
\makeatother

\newcommand{\todofigure}[1]{\missingfigure{#1}\todo{#1}}


\usepackage{pdfpages}
\newcommand{\insertpaper}[3]%
           { % Syntax: \insertpaper{title}{label}{File}
             % Requires: tocloft hyperref pdfpages
            \includepdf[addtotoc={1,section,1,#1,#2}]{#3}
            \includepdf[pages=2-last]{#3}
           }

%Inkscape inclusions
\usepackage{bashful}
\usepackage{import}
%\usepackage{ifplatform}
\usepackage{etoolbox}

\newbool{canInkscape} %default false
\setbool{canInkscape}{true}

\newcommand{\includesvg}[1]{
    \ifbool{canInkscape}{
        \immediate\write18{sh tools/inkscapeSVG2PDFtex.sh #1}
        }{}%
\IfFileExists{#1.pdf_tex}{\textsf{\input{#1.pdf_tex}}}{\textsf{#1.pdf\_tex is missing. inkscapeSVG2PDFtex.sh likely failed.  \\ \rule{5cm}{5cm}}}
}


%fig macros
\newcommand{\figuremacro}[3]{
	\begin{figure}[b]
		\centering
		\includegraphics[width=1\textwidth]{#1}
		\caption[#2]{{#2} - #3}
		\label{#1}
	\end{figure}
}

% insert a centered figure with caption and description AND WIDTH
% parameters 1:filename, 2:title, 3:description and label, 4: textwidth
% textwidth 1 means as text, 0.5 means half the width of the text
\newcommand{\figuremacroW}[4]{
	\begin{figure}
		\centering
		\includegraphics[width=#4\textwidth]{#1}
		\caption[#2]{#3}
		\label{fig:#1}
	\end{figure}
}

\newcommand{\figuremacroWp}[5]{
	\begin{figure}[#5]
		\centering
		\includegraphics[width=#4\textwidth]{#1}
		\caption[#2]{#3}
		\label{fig:#1}
	\end{figure}
}

% insert a centered subfigure with caption and description AND WIDTH
% parameters 1:filename, 2:title, 3:description and label, 4: textwidth
% textwidth 1 means as text, 0.5 means half the width of the text
%\newcommand{\figuremacroSUB}[7]{
%	\begin{figure}[tbh]
%	   \centering
%		\subfloat[]{\label{#2}\includegraphics[width=#4\textwidth]{#2}}
%		\quad
%		\subfloat[]{\label{#5}\includegraphics[width=#7\textwidth]{#5}}
%	   \caption[#1]{{#1} - {\it a)} #3  {\it b)} #6}
%	   \label{#1}
%	\end{figure}
%}

%1:title 2:file1 3:caption1 4:width1 5:file2 6:caption2 7:width2
\newcommand{\figuremacroSUB}[7]{
\begin{figure}[tbh]
\centering
     % subfig 1
	 \begin{subfigure}[b]{#4\textwidth}
	     \centering
         \includegraphics[width=\textwidth]{#2}
         \caption{}
         \label{fig:#2}
	 \end{subfigure} 
	 % subfig 2
	 \begin{subfigure}[b]{#7\textwidth}
	     \centering
         \includegraphics[width=\textwidth]{#5}
         \caption{}
         \label{fig:#5}
	 \end{subfigure}
	 	 
	 \vspace{-0.2cm}
	 \caption[#1]{{\it a)} #3  {\it b)} #6}
	 \label{fig:#1}
\end{figure}
}

\newcommand{\figuremacroSUBp}[8]{
\begin{figure}[#8]
\centering
     % subfig 1
	 \begin{subfigure}[b]{#4\textwidth}
	     \centering
         \includegraphics[width=\textwidth]{#2}
         \caption{}
         \label{fig:#2}
	 \end{subfigure} 
	 % subfig 2
	 \begin{subfigure}[b]{#7\textwidth}
	     \centering
         \includegraphics[width=\textwidth]{#5}
         \caption{}
         \label{fig:#5}
	 \end{subfigure}
	 
	 \vspace{-0.2cm}
	 \caption[#1]{{\it a)} #3  {\it b)} #6}
	 \label{fig:#1}
\end{figure}
}


%\figuremacroSUB{title}{file1}{caption1}{size1}{file2}{caption2}{size2}

% inserts a figure with wrapped around text; only suitable for NARROW figs
% o is for outside on a double paged document; others: l, r, i(inside)
% text and figure will each be half of the document width
% note: long captions often crash with adjacent content; take care
% in general: above 2 macro produce more reliable layout
\newcommand{\figuremacroN}[3]{
	\begin{wrapfigure}{o}{0.5\textwidth}
		\centering
		\includegraphics[width=0.48\textwidth]{#1}
		\caption[#2]{{\small\textbf{#2} - #3}}
		\label{#1}
	\end{wrapfigure}
}

% predefined commands by Harish
\newcommand{\PdfPsText}[2]{
  \ifpdf
     #1
  \else
     #2
  \fi
}

\newcommand{\IncludeGraphicsH}[3]{
  \PdfPsText{\includegraphics[height=#2]{#1}}{\includegraphics[bb = #3, height=#2]{#1}}
}

\newcommand{\IncludeGraphicsW}[3]{
  \PdfPsText{\includegraphics[width=#2]{#1}}{\includegraphics[bb = #3, width=#2]{#1}}
}

\newcommand{\InsertFig}[3]{
  \begin{figure}[!htbp]
    \begin{center}
      \leavevmode
      #1
      \caption{#2}
      \label{#3}
    \end{center}
  \end{figure}
}

%% fix biblatex's fullcite to use maxbibnames
%\makeatletter
%\DeclareCiteCommand{\fullcite}
%  {\defcounter{maxnames}{\blx@maxbibnames}%
%    \usebibmacro{prenote}}
%  {\usedriver
%     {\DeclareNameAlias{sortname}{default}}
%     {\thefield{entrytype}}}
%  {\multicitedelim}
%  {\usebibmacro{postnote}}
%\DeclareCiteCommand{\footfullcite}[\mkbibfootnote]
%  {\defcounter{maxnames}{\blx@maxbibnames}%
%    \usebibmacro{prenote}}
%  {\usedriver
%     {\DeclareNameAlias{sortname}{default}}
%     {\thefield{entrytype}}}
%  {\multicitedelim}
%  {\usebibmacro{postnote}}
%\makeatother


%%% Local Variables: 
%%% mode: latex
%%% TeX-master: "~/Documents/LaTeX/CUEDThesisPSnPDF/thesis"
%%% End: 



%: ----------------------------------------------------------------------
%:                  TITLE PAGE: name, degree,..
% -----------------------------------------------------------------------
% below is to generate the title page with crest and author name

% Document metadata - define once, use everywhere
% \def\thesistitle{A new approach to classical spectroscopy.}
% \def\thesisauthor{Christopher H. Betters}
% \def\thesisauthoremail{chris@chrisbetters.com}
% \def\thesiskeywords{Astrophotonics; Diffraction-limited spectroscopy; PhD Thesis}
% \def\thesislogo{0_frontmatter/figures/usydlogo_bw.pdf}
% \def\thesiscollege{Faculty of Science}
% \def\thesisuniversity{The University of Sydney}
% \def\thesisdegree{Doctor of Philosophy}


% Document setup using dynamic variables
\title{Your amazing thesis title.}

% ----------------------------------------------------------------------
% The section below defines www links/email for author and institutions
% They will appear on the title page of the PDF and can be clicked
\author{First Lastname}
\def\thesisauthoremail{First.Lastname@sydney.edu.au}
\def\thesiskeywords{Science; PhD Thesis}


\collegeordept{Faculty of Science}
\university{The University of Sydney}
\crest{
  \includegraphics[width=6cm]{0_frontmatter/figures/usydlogo_bw.pdf}
}

\degree{Doctor of Philosophy}
\degreedate{\monthyeardate\today}
\degreeyear{\the\year}
% ----------------------------------------------------------------------

%PDF output configuration using dynamic variables
\hypersetup{pdfinfo={
Title={\thetitle},
Author={\theauthor\space\thesisauthoremail},
Keywords={\thesiskeywords}
},
colorlinks,
citecolor=SchoolColor,
filecolor=black,
linkcolor=black,
urlcolor=SchoolColor}

       
% turn of those nasty overfull and underfull hboxes
\hbadness=10000
\hfuzz=50pt

%: --------------------------------------------------------------
%:                         REFERENCES
%: --------------------------------------------------------------
%\addbibresource{bib/references.bib} % must have .bib
%tex
%\addbibresource{bib/allofpapers.bib}
%\addbibresource{bib/referencesSPIE2012-1.bib}
%\addbibresource{bib/referencesSPIE2012-2.bib}

\addbibresource{bib/thesisRefs.bib}
\addbibresource{bib/fake_publications.bib}

\makeglossaries

%\includeonly{0_frontmatter/abstract,0_frontmatter/list_of_publications,0_frontmatter/acknowledgement,0_frontmatter/declaration}
% \includeonly{1_introduction/overview}
% \includeonly{2/macros}
% \includeonly{3/figures}


%: --------------------------------------------------------------
%:                  FRONT MATTER: dedications, abstract,..
% --------------------------------------------------------------
\begin{document}

% sets line spacing
\renewcommand\baselinestretch{1.2}
\baselineskip=18pt plus1pt

\setcounter{secnumdepth}{2} % organisational level that receives a numbers
\setcounter{tocdepth}{1}    % print table of contents for level 3

%: ----------------------- generate cover page ------------------------
\ifthesisisdraft
	\frontmatter
	\listoftodos
	%\listoffigures
\else
	\maketitle  % command to print the title page with above variables
	\copyrightpage
	
	%: ----------------------- cover page back side ------------------------
	% Your research institution may require reviewer names, etc.
	% This cover back side is required by Dresden Med Fac; uncomment if needed.
	%
	%\newpage
	%\vspace{10mm}
	%1. Reviewer: Name
	%
	%\vspace{10mm}
	%2. Reviewer:
	%
	%\vspace{20mm}
	%Day of the defense:
	%
	%\vspace{20mm}
	%\hspace{70mm}Signature from head of PhD committee:
	
	
	%: ----------------------- abstract ------------------------
	
	% Your institution may have specific regulations if you need an abstract and where it is to be placed in the document. The default here is just after title.
	
	\include{0_frontmatter/abstract}
	
	% The original template provides and abstractseparate environment, if your institution requires them to be separate. I think it's easier to print the abstract from the complete thesis by restricting printing to the relevant page.
	% \begin{abstractseparate}
	%   \input{Abstract/abstract}
	% \end{abstractseparate}
	
	
	%: ----------------------- tie in front matter ------------------------
	
	\frontmatter
	%\include{0_frontmatter/dedication}
	% Thesis Acknowledgements ------------------------------------------------


%\begin{acknowledgementslong} %uncommenting this line, gives a different acknowledgements heading
\begin{acknowledgements}      %this creates the heading for the acknowlegments
\lipsum[1-3]
\end{acknowledgements}
%\end{acknowledgmentslong}

% ------------------------------------------------------------------------



	
% Thesis statement of originality -------------------------------------

% Depending on the regulations of your faculty you may need a declaration like the one below. This specific one is from the medical faculty of the university of Dresden.

\begin{declaration}        %this creates the heading for the declaration page

This thesis describes work carried out in the Sydney Institute for Astronomy, within the School of Physics, University of Sydney, between February 2011 and March 2015. The work presented in this thesis is, to the best of my knowledge and belief, original except as acknowledged in the text. I hereby declare that I have not submitted this material, either in full or in part, for a degree or diploma at this university or any other institution.

\vspace{10mm}

\includegraphics[height=1.5cm]{sig_trans} \\
\theauthor{}\\
%\today
\date{April 21st, 2015}
\end{declaration}


% ----------------------------------------------------------------------
	\chapter{Publications And Contribution}
%\addcontentsline{toc}{chapter}{Publications And Contribution}

\small

\section*{Journal Papers}
\begin{itemize}%\setlength\itemsep{-0.3em}
	\item \fullciteandexclude{yourname2024nature}
	\\ Contents included throughout \chap{DesignPrinciples} and \sect{figmacros}.
	
	\item \fullciteandexclude{yourname2023mnras}
	\\ Contents included as part of \chap{intro} and \sect{crossref}.
	
	\item \fullciteandexclude{expert2023review}
	\\ Contents included as part of \chap{intro}.
	
	
	\item \fullciteandexclude{adams2022pasp}
	\\ Contents included as part of \sect{advancedfig}.
	
\end{itemize}

\section*{Refereed Conference Papers}
\begin{itemize}%\setlength\itemsep{-0.3em}
	\item \fullciteandexclude{yourname2024spie1}
	\\Contents included as part of \sect{advancedfig} and \sect{placeholder}.
	
	\item \fullciteandexclude{yourname2023spie2}
	\\Contents included as part of \sect{sampletable} and \sect{crosschap}.
	
	\item \fullciteandexclude{lee2023spie3}
	
	\item \fullciteandexclude{yourname2022spie4}
\end{itemize}

\section*{Conference Presentations}
\begin{itemize}%\setlength\itemsep{-0.3em}
	\item \fullciteandexclude{yourname2024photwest}
	
	\item \fullciteandexclude{yourname2023labchar}
\end{itemize}

\section*{Awards and Recognition}
\begin{itemize}%\setlength\itemsep{-0.3em}
	\item Best Student Paper Award, SPIE Astronomical Telescopes + Instrumentation (2023)
	\item University of Sydney Postgraduate Research Scholarship (2021--2024)
	\item Optical Society Student Travel Grant, Photonics West (2023)
\end{itemize}

\section*{Funding}
This research was supported by an Australian Government Research Training Program (RTP) Scholarship.

% See https://sydneyuni.service-now.com/sm?id=kb_article_view&sysparm_article=KB0035308
\section*{Use of Gen AI}
See \url{https://sydneyuni.service-now.com/sm?id=kb_article_view&sysparm_article=KB0035308} for more information.
\subsection*{When Gen AI has not been used}
No content produced by generative AI tools has been used in the preparation of this thesis

\subsection*{When Gen AI has been used for minimal copyediting}

During the preparation of the thesis the author used [Name of tool or service] for the purposes of [e.g. text enhancement. The use of this generative AI tool includes LIST EXAMPLES OF WHERE TEXT WAS ADJUSTED (e.g. paraphrasing, sentence structure, spelling, etc)].  The author confirms that where text was modified by generative AI, the content was reviewed for possible errors, inaccuracies, and bias. The author takes full responsibility for the submitted thesis and ensures the work is their own and has used generative AI within the parameters of use (refer to the University of Sydney generative AI guide for researchers).

\subsection*{When Gen AI has been used as part of the integral research design}

“During the preparation of this thesis, [Name of tool or service] was used as an integral part of the research design, which will be included within the methodology section for [e.g. Chapters X,Y,Z]. In text citation should be included for any section of text and/or result figures that were generated by a generative AI tool.  The generative AI tool was not used to enhance or change text. The author takes full responsibility for the submitted thesis and confirms the work is their own and has used generative AI in accordance with University guidelines and policies (refer to the University of Sydney generative AI guide for researchers).”

\subsection*{When Gen AI has been used for multiple uses and/or moderate use throughout the thesis (integral research, moderate copyediting, etc)}

During the preparation of this thesis the author used [NAME TOOL / SERVICE], in order to [Provide appendix reference here - LIST REASONS in appendix]. The author confirms that where text was modified by generative AI, the content was reviewed for possible errors, inaccuracies, and bias. The author takes full responsibility for the submitted thesis and ensures the work is their own and has used generative AI in accordance with University  guidelines and policies .(refer to the University of Sydney generative AI guide for researchers).



\citereset


%: ----------------------- contents ------------------------

\tableofcontents            % print the table of contents
% levels are: 0 - chapter, 1 - section, 2 - subsection, 3 - subsubsection

%%: ----------------------- list of figures/tables ------------------------
\listoffigures	% print list of figures
\listoftables  % print list of tables

\fi
%: --------------------------------------------------------------
%: --------------------------------------------------------------
%:                  MAIN DOCUMENT SECTION
% --------------------------------------------------------------
%: --------------------------------------------------------------

% the main text starts here with the introduction, 1st chapter,...
\mainmatter
%\RaggedRight
%\setlength{\parskip}{12pt plus 0.5ex minus 0.5ex}

\renewcommand{\chaptername}{} % uncomment to print only "1" not "Chapter 1"

%: ----------------------- subdocuments ------------------------

% Parts of the thesis are included below. Rename the files as required.
% But take care that the paths match. You can also change the order of appearance by moving the include commands.

\graphicspath{{0_overview/figures/}{1_introduction/figures/}{2/figures/}{3/figures/}{4/figures/}{5/figures/}{6/figures/}{7/figures/}}
% this file is called up by thesis.tex
% content in this file will be fed into the main document

%\newacronym{<label>}{<abbrv>}{<full>}


%\newglossaryentry{naiive}
%{
%  name=na\"{\i}ve,
%  description={is a French loanword (adjective, form of naïf)
%               indicating having or showing a lack of experience,
%               understanding or sophistication},
%  sort=naive
%}


% todo Glossary Shortcut
%% locaization stuuf

\newif\ifamerican

\americanfalse

\ifamerican
  \newcommand{\fibre}{fiber\xspace}
  \newcommand{\fibres}{fibers\xspace}
\else
  \newcommand{\fibre}{fibre\xspace}
  \newcommand{\fibres}{fibres\xspace}
\fi


%\newacronym{smf}{SMF}{single-mode \fibre\xspace}
\newglossaryentry{smf}
{
  name={SMF},
  description={single-mode \fibre},
  first={single-mode \fibre (\glsentrytext{smf})},
  plural={SMFs},
  descriptionplural={single-mode \fibres},
  firstplural={single-mode \fibres (\glsentryplural{smf})}
}
\newcommand{\smf}{\gls{smf}\xspace}
\newcommand{\smfs}{\glspl{smf}\xspace}


\newacronym{sm}{SM}{single-mode\xspace}
\newcommand{\sm}{\gls{sm}\xspace}

\newacronym{mm}{MM}{multi-mode\xspace}
\newcommand{\MM}{\gls{mm}\xspace}

%\newacronym{mmf}{MMF}{multi-mode \fibre}
\newglossaryentry{mmf}
{
  name={MMF},
  description={multi-mode \fibre},
  first={multi-mode \fibre (\glsentrytext{mmf})},
  plural={MMFs},
  descriptionplural={multi-mode \fibres},
  firstplural={multi-mode \fibres (\glsentryplural{mmf})}
}
\newcommand{\MMF}{\gls{mmf}\xspace}
\newcommand{\MMFs}{\glspl{mmf}\xspace}

\newacronym{pimms}{PIMMS}{photonic integrated multi-mode spectrograph}
\newcommand{\pimms}{\gls{pimms}\xspace}

\newacronym{mcf}{MCF}{multi-core fibre}
\newcommand{\mcf}{\gls{mcf}\xspace}

\newacronym{fwhm}{FWHM}{full width half maximum}
\newcommand{\fwhm}{\gls{fwhm}\xspace}

\newacronym{ifu}{IFU}{integral field unit}
\newcommand{\ifu}{\gls{ifu}\xspace}

\newacronym{uli}{ULI}{ultrafast laser inscription}
\newcommand{\uli}{\gls{uli}\xspace}

\newacronym{vph}{VPH}{volume-phase holographic}
\newcommand{\vph}{\gls{vph}\xspace}

\newacronym{ft}{FT}{Fourier transform}
\newcommand{\ft}{\gls{ft}\xspace}

\newacronym{ar}{AR}{anti-reflective}
\newcommand{\AR}{\gls{ar}\xspace}

\newacronym{fts}{FTS}{Fourier transform spectrometer}
\newcommand{\fts}{\gls{fts}\xspace}
\newcommand{\ftss}{\glspl{fts}\xspace}

\newacronym{mos}{MOS}{multi-object spectrograph}
\newcommand{\mos}{\gls{mos}\xspace}
\newcommand{\moss}{\glspl{mos}\xspace}

\newacronym{psf}{PSF}{point spread function}
\newcommand{\psf}{\gls{psf}\xspace}
\newcommand{\psfs}{\glspl{psf}\xspace}

\newacronym{lsf}{LSF}{line spread function}
\newcommand{\lsf}{\gls{psf}\xspace}

\newacronym{snr}{SNR}{single-to-noise ratio}
\newcommand{\snr}{\gls{snr}\xspace}

\newacronym{rms}{RMS}{root mean square}
\newcommand{\rms}{\gls{rms}\xspace}

\newacronym{fsr}{FSR}{free spectral range}
\newcommand{\fsr}{\gls{fsr}\xspace}

\newacronym{na}{NA}{numerical aperture}
\newcommand{\NA}{\gls{na}\xspace}

\newacronym{cots}{COTS}{commercial off-the-shelf}
\newcommand{\COTS}{\gls{cots}\xspace}

\newacronym{pcb}{PCB}{printed circuit board}
\newcommand{\pcb}{\gls{pcb}\xspace}
\newcommand{\pcbs}{\glspl{pcb}\xspace}

\newacronym{pop}{POP}{physical optics propagation}
\newcommand{\pop}{\gls{pop}\xspace}

\newacronym{pcf}{PCF}{photonic crystal fibre}
\newcommand{\pcf}{\gls{pcf}\xspace}

\newacronym{efl}{EFL}{effective focal length}
\newcommand{\efl}{\gls{efl}\xspace}
\newcommand{\efls}{\glspl{efl}\xspace}

\newacronym{thar}{ThAr}{Thorium-Argon}
\newcommand{\thar}{\gls{thar}\xspace}

\newacronym{fp}{FP}{Fabry-P\'erot}
\newcommand{\FabryPerot}{\gls{fp}\xspace}

\newacronym{ao}{AO}{adaptive optics}
\newcommand{\AO}{\gls{ao}\xspace}

\newacronym{tec}{TEC}{thermoelectric cooler}
\newcommand{\tec}{\gls{tec}\xspace}

\newacronym{ptv}{PTV}{peak to valley}
\newcommand{\ptv}{\gls{ptv}\xspace}

\newacronym{dft}{DFT}{direct Fourier transfrom}
\newcommand{\DFT}{\gls{dft}\xspace}

\newacronym{fft}{FFT}{fast Fourier transfrom}
\newcommand{\fft}{\gls{fft}\xspace}

\newacronym{adu}{ADU}{analogue-to-digital units}
\newcommand{\ADU}{\gls{adu}\xspace}

\newglossaryentry{pl}{
    name={PL},
    description={photonic lantern; a multi-mode to single-mode converter},
    first={photonic lantern (PL)}
}
\newcommand{\lantern}{\gls{pl}\xspace}
\newcommand{\lanterns}{\glspl{pl}\xspace}
\newcommand{\pl}{\gls{pl}\xspace}
\newcommand{\pls}{\glspl{pl}\xspace}

\newglossaryentry{awg}{
    name={AWG},
    description={arrayed waveguide grating; a device with a series single mode waveguides where each waveguide has increases in length by a fixed phase offset, producing a high order diffraction/dispersion},
    first={arrayed waveguide grating (AWG)}
}
\newcommand{\awg}{\gls{awg}\xspace}
\newcommand{\awgs}{\glspl{awg}\xspace}


\newglossaryentry{ips}{
    name={IPS},
    description={integrated photonic spectrograph; spectrograph composed of integrated photonics technologies, i.e. \uli photonics lantern fused to \awg},
    first={integrated photonic spectrograph (IPS)}
}
\newcommand{\ips}{\gls{ips}\xspace}

\newglossaryentry{leo}{
    name={LEO},
    description={low earth orbit is considered to be between \SIrange{160}{2000}{\kilo\meter}},
    first={low earth orbit (LEO)}
}
\newcommand{\leo}{\gls{leo}\xspace}


\newglossaryentry{pimms0}{
    name={PIMMS\#0},
    description={Hybrid (photonic and bulk optic) version of PIMMS},
}
\newcommand{\pimmsZ}{\gls{pimms0}\xspace}

\newglossaryentry{pimms1}{
    name={PIMMS\#1},
    description={Fully photonic version of PIMMS},
}
\newcommand{\pimmsOne}{\gls{pimms1}\xspace}

\newglossaryentry{etendue}{
    name={\'{e}tendue},
    symbol={$A\Omega$},
    description={geometric characterisation of an optical systems ability to accept light. Defined as product of area and solid angle of acceptance}
    }
\newcommand{\etendue}{\gls{etendue}\xspace}
\newcommand{\AOm}{\glssymbol{etendue}\xspace}

\newglossaryentry{throughput}{
    name={throughput},
    description={ratio of photons detected to photons collected.},
    first={throughput}
}
\newcommand{\throughput}{\gls{throughput}\xspace}

\newglossaryentry{resolvingpower}{
    name={resolving power},
    symbol={{$\mathcal{R}$}},
    description={quantitative measure of a spectrographs ability to resolve neighbouring spectral features. Defined as $\mathcal{R}=\lambda/\Delta\lambda$},
    first={resolving power ($\mathcal{R}$)}
}
\newcommand{\R}{\glssymbol{resolvingpower}}
\newcommand{\resolvingpower}{\gls{resolvingpower}\xspace}
\newcommand{\Rmath}{\mathcal{R}}
\newcommand{\Rdef}{$\lambda/\Delta\lambda$}


\newglossaryentry{spectralresolution}{
    name={spectral resolution},
    symbol={$\Delta\lambda$},
    description={measure of the smallest wavelength difference resolvable at wavelength $\lambda$. Typically measured as the \fwhm of a spectral feature}
}
\newcommand{\dlam}{\glssymbol{spectralresolution}\xspace}
\newcommand{\spectralresolution}{\gls{spectralresolution}\xspace}


\newglossaryentry{f-ratio}{
    name={focal ratio},
    symbol={$f/\#$},
    description={ratio of focal length and aperture/pupil diameter, defined as $f/\# = F/D$}
}
\newcommand{\fnum}{\glssymbol{f-ratio}\xspace}
\newcommand{\fratio}{\gls{f-ratio}\xspace}

\newglossaryentry{seeing}{
    name={seeing},
    description={the blurring of a telescope image due to atmospheric turbulence and varying the optical refractive index}
}
\newcommand{\seeing}{\gls{seeing}\xspace}

\newglossaryentry{swir}{
	name={SWIR},
    first={short-wave infrared (SWIR)},
    description={light with wavelengths from 1\um to 3\um}
}
\newcommand{\swir}{\gls{swir}\xspace}

\newglossaryentry{ir}{
	name={IR},
    first={infrared (IR)},
    description={light with wavelengths from 0.7\um to 100\um, but often used in context to indicate smaller regions.}
}
\newcommand{\ir}{\gls{ir}\xspace}

\newglossaryentry{nir}{
	name={NIR},
    first={near-infrared (NIR)},
    description={light with wavelength 0.7\um to 1\um, or edge of human vision to the silicon sensitivity cutoff}
}
\newcommand{\nir}{\gls{nir}\xspace}

\newglossaryentry{uv}{
	name={UV},
    first={ultraviolet (UV)},
    description={light with wavelength 10~nm to 400~nm.}
}
\newcommand{\uv}{\gls{uv}\xspace}


\newglossaryentry{mfd}{
    name={MFD},
    description={diameter of the intensity profile of light at the out put of \fibre. Normally measured at the width at the $1/e^2$ intensity level},
    first={mode-field diameter (MFD)}
}
\newcommand{\mfd}{\gls{mfd}\xspace}

%\newglossaryentry{}{
%    name={},
%    symbol={$$},
%    description={},
%	 first={near-infrared (NIR)}
%}
%\newcommand{}{\glssymbol{}\xspace}




 % include now for macros, print at end

\chapter{Template Overview and Usage Guide}
\label{chap:overview}
\newrefsegment
\glsresetall

This chapter provides a comprehensive guide for using this LaTeX thesis template. The template is designed for University of Sydney PhD theses and includes advanced features for figure management, bibliography handling, cross-referencing, and document customization.

\section{Getting Started}
\label{sect:getting-started}

To begin using this template, you need to customize several key files with your specific thesis information. This section outlines the essential modifications required.

\subsection{Document Metadata Configuration}
\label{sect:metadata-config}

The primary customization occurs in \texttt{preamble.tex}, where you define your thesis metadata. Update the following variables:

\begin{lstlisting}[language=TeX,caption={Document metadata in preamble.tex}]
% Document setup using dynamic variables
\title{Your amazing thesis title.}
\author{First Lastname}
\def\thesisauthoremail{First.Lastname@sydney.edu.au}
\def\thesiskeywords{Science; PhD Thesis}

\collegeordept{Faculty of Science}
\university{The University of Sydney}
\degree{Doctor of Philosophy}
\degreedate{\monthyeardate\today}
\degreeyear{\the\year}
\end{lstlisting}

These variables automatically populate the title page, PDF metadata, and hyperlink information throughout your document.

\subsection{Draft Mode Configuration}
\label{sect:draft-mode}

The template supports draft mode for faster compilation during development. In \texttt{thesis.tex}, you can toggle draft mode:

\begin{lstlisting}[language=TeX,caption={Draft mode toggle in thesis.tex}]
% Toggle draft mode - figures show as placeholder boxes
\thesisisdrafttrue  % Enable draft mode
% \thesisisdraftfalse % Disable draft mode for final version
\end{lstlisting}

When draft mode is enabled, all figures display as placeholder boxes, significantly reducing compilation time.

\section{Bibliography Management}
\label{sect:bibliography}

The template uses the biblatex package with biber backend for sophisticated bibliography management.

\subsection{Bibliography Files}
\label{sect:bib-files}

Bibliography sources are configured in \texttt{preamble.tex}:

\begin{lstlisting}[language=TeX,caption={Bibliography configuration}]
\addbibresource{bib/thesisRefs.bib}
\addbibresource{bib/fake_publications.bib}
\end{lstlisting}

Add your own \texttt{.bib} files to the \texttt{bib/} directory and include them here.

\subsection{Citation Commands}
\label{sect:citations}

The template uses biblatex citation commands. Key commands include:

\begin{itemize}
    \item \texttt{\textbackslash autocite\{key\}} - Automatic citation format: \autocite{yourname2024nature}
    \item \texttt{\textbackslash textcite\{key\}} - In-text citation: \textcite{yourname2024nature}
    \item \texttt{\textbackslash parencite\{key\}} - Parenthetical citation: \parencite{yourname2024nature}
    \item \texttt{\textbackslash fullcite\{key\}} - Full citation in text
    \item \texttt{\textbackslash fullciteandexclude\{key\}} - Full citation excluded from bibliography
\end{itemize}

\subsection{Compilation Sequence}
\label{sect:compilation}

For proper bibliography processing, use this compilation sequence:

\begin{lstlisting}[language=bash,caption={Complete compilation sequence}]
pdflatex thesis.tex
biber thesis
pdflatex thesis.tex
pdflatex thesis.tex  # Second run for cross-references
\end{lstlisting}

\section{Figure Management}
\label{sect:figure-management}

The template provides powerful figure insertion macros defined in \texttt{Latex/Macros/MacroFile1.tex}.

\subsection{Figure Insertion Macros}
\label{sect:figure-macros}

Key figure macros include:

\begin{description}
    \item[\texttt{\textbackslash figuremacroW\{filename\}\{label\}\{caption\}\{width\}}] Single figure with width control
    \item[\texttt{\textbackslash figuremacroWp\{filename\}\{label\}\{caption\}\{width\}\{position\}}] Single figure with position control
    \item[\texttt{\textbackslash figuremacroSUB\{label\}\{file1\}\{caption1\}\{width1\}\{file2\}\{caption2\}\{width2\}}] Side-by-side figures
    \item[\texttt{\textbackslash figuremacroSUBp\{...\}\{position\}}] Side-by-side with position control
\end{description}

These macros automatically handle figure placement, labeling, and formatting consistency.

\subsection{Cross-Referencing System}
\label{sect:cross-ref-system}

The template provides consistent cross-referencing macros:

\begin{description}
    \item[\texttt{\textbackslash fig\{label\}}] Short figure reference (Fig.~X.X)
    \item[\texttt{\textbackslash Fig\{label\}}] Long figure reference (Figure~X.X)
    \item[\texttt{\textbackslash eqn\{label\}}] Short equation reference (Eqn.~X.X)
    \item[\texttt{\textbackslash Eqn\{label\}}] Long equation reference (Equation~X.X)
    \item[\texttt{\textbackslash sect\{label\}}] Section reference (Sec.~X.X)
    \item[\texttt{\textbackslash chap\{label\}}] Chapter reference (Chap.~X)
    \item[\texttt{\textbackslash tabl\{label\}}] Table reference (Table~X.X)
\end{description}

\section{Chapter Organization}
\label{sect:chapter-organization}

\subsection{Chapter Structure}
\label{sect:chapter-structure}

Each chapter should follow this structure:

\begin{lstlisting}[language=TeX,caption={Standard chapter template}]
\chapter{Chapter Title}
\label{chap:chapter-label}
\newrefsegment  % Start new reference segment
\glsresetall    % Reset glossary entries

% Chapter content here
\end{lstlisting}

The \texttt{\textbackslash newrefsegment} command enables per-chapter bibliography sections if needed, while \texttt{\textbackslash glsresetall} resets glossary entry formatting.

\subsection{Adding New Chapters}
\label{sect:adding-chapters}

To add a new chapter:

\begin{enumerate}
    \item Create a new \texttt{.tex} file in an appropriate directory
    \item Add the chapter structure shown above
    \item Include the file in \texttt{thesis.tex} using \texttt{\textbackslash include\{path/to/chapter\}} (without .tex extension)
    \item Update the table of contents if necessary
\end{enumerate}

\subsection{Selective Compilation with \texttt{\textbackslash includeonly}}
\label{sect:includeonly}

When working on a large thesis, you may want to compile only specific chapters to save time. The template supports this through LaTeX's \texttt{\textbackslash includeonly} command.

\subsubsection{Using \texttt{\textbackslash includeonly} in Overleaf}

In the main \texttt{thesis.tex} file, you'll find commented examples:

\begin{lstlisting}[language=TeX,caption={includeonly examples in thesis.tex}]
%\includeonly{0_frontmatter/abstract,0_frontmatter/list_of_publications}
% \includeonly{1_introduction/overview}
% \includeonly{2/macros}
% \includeonly{3/figures}
\end{lstlisting}

To compile only specific chapters:

\begin{enumerate}
    \item \textbf{Uncomment} the \texttt{\textbackslash includeonly} line for your desired chapters
    \item \textbf{Ensure only one} \texttt{\textbackslash includeonly} command is active at a time
    \item \textbf{Recompile} in Overleaf - only the specified chapters will be processed
    \item \textbf{Comment out} the \texttt{\textbackslash includeonly} line when you want to compile the full document
\end{enumerate}

\subsubsection{Benefits for Overleaf Users}

\begin{itemize}
    \item \textbf{Faster compilation}: Large theses with many figures compile much faster when working on individual chapters
    \item \textbf{Reduced timeout risk}: Overleaf has compilation time limits; selective compilation helps stay within them
    \item \textbf{Preserved cross-references}: LaTeX maintains references to other chapters even when they're not compiled
    \item \textbf{Efficient editing}: Focus on specific sections without waiting for the entire document to compile
\end{itemize}

\subsubsection{Important Notes}

\begin{itemize}
    \item Use \texttt{\textbackslash include\{path/filename\}} not \texttt{\textbackslash input\{path/filename\}} for chapters you want to selectively compile
    \item The file path in \texttt{\textbackslash includeonly} must exactly match the path in \texttt{\textbackslash include}
    \item Cross-references to excluded chapters will show as ``??'' until you compile the full document
    \item Always do a final compilation with \texttt{\textbackslash includeonly} commented out before submission
\end{itemize}

\subsubsection{Example Workflow in Overleaf}

\begin{enumerate}
    \item Working on methodology chapter: Uncomment \texttt{\textbackslash includeonly\{2/macros\}}
    \item Make edits to your methodology content
    \item Compile quickly to see changes (faster than full document)
    \item When ready to see full document with all references: Comment out \texttt{\textbackslash includeonly}
    \item Final compile processes entire thesis with all cross-references resolved
\end{enumerate}

\section{Publications and Awards}
\label{sect:publications-awards}

\subsection{List of Publications}
\label{sect:list-publications}

The template includes \texttt{0\_frontmatter/list\_of\_publications.tex} for showcasing your academic contributions. This file uses:

\begin{lstlisting}[language=TeX,caption={Publication listing format}]
\section*{Journal Papers}
\begin{itemize}
    \item \fullciteandexclude{paper-key}
    \\ Contents included throughout \chap{chapter-label} and \sect{section-label}.
\end{itemize}
\end{lstlisting}

The \texttt{\textbackslash fullciteandexclude\{key\}} command displays the full citation but excludes it from the main bibliography to avoid duplication.

\subsection{Connecting Publications to Content}
\label{sect:connecting-publications}

For each publication, specify which chapters and sections incorporate that work. This demonstrates the connection between your published research and thesis content.

\section{Advanced Features}
\label{sect:advanced-features}

\subsection{Glossaries and Acronyms}
\label{sect:glossaries}

The template supports glossaries through the \texttt{glossaries} package. Define terms in glossary files and compile with:

\begin{lstlisting}[language=bash]
makeglossaries thesis
\end{lstlisting}

\subsection{Code Listings}
\label{sect:code-listings}

The template includes support for code listings with syntax highlighting:

\begin{lstlisting}[language=TeX,caption={Code listing example}]
\begin{lstlisting}[language=Python,caption={Python example}]
def hello_world():
    print("Hello, World!")
\end{lstlisting}

\subsection{Mathematical Content}
\label{sect:math-content}

The template loads \texttt{amsmath} and related packages for comprehensive mathematical typesetting. Use standard LaTeX math environments:

\begin{equation}
    \nabla \times \mathbf{E} = -\frac{\partial \mathbf{B}}{\partial t}
    \label{eqn:maxwell-faraday}
\end{equation}

Reference equations using \texttt{\textbackslash eqn\{maxwell-faraday\}} or \texttt{\textbackslash Eqn\{maxwell-faraday\}}.

\section{Customization Options}
\label{sect:customization}

\subsection{Color Scheme}
\label{sect:color-scheme}

University colors are defined in \texttt{preamble.tex}:

\begin{lstlisting}[language=TeX,caption={Color definitions}]
\definecolor{SchoolColor}{RGB}{230, 70, 38} % Masterbrand red
\definecolor{chaptergrey}{RGB}{230, 70, 38}
\definecolor{midgrey}{RGB}{66, 66, 66}
\end{lstlisting}

Modify these values to match your requirements while maintaining university branding guidelines.

\subsection{Font Configuration}
\label{sect:font-config}

The template uses Computer Modern fonts by default. Font modifications should be made carefully to maintain readability and university requirements.

\subsection{Page Layout}
\label{sect:page-layout}

Page layout parameters are defined in \texttt{Latex/Classes/PhDthesisPSnPDF.cls}. Modify margins and spacing only if required by university guidelines.

\section{Troubleshooting}
\label{sect:troubleshooting}

\subsection{Common Issues}
\label{sect:common-issues}

\begin{description}
    \item[Bibliography not appearing] Ensure you run \texttt{biber} after \texttt{pdflatex}
    \item[Cross-references showing as ??] Run \texttt{pdflatex} multiple times for cross-reference resolution
    \item[Figures not displaying] Check file paths and ensure draft mode is disabled
    \item[Glossary not generating] Run \texttt{makeglossaries} and recompile
\end{description}

\subsection{Compilation Errors}
\label{sect:compilation-errors}

For complex documents, compilation errors may occur. Common solutions:

\begin{itemize}
    \item Clear auxiliary files (\texttt{.aux}, \texttt{.bbl}, \texttt{.blg}, etc.) and recompile
    \item Check for undefined references or missing files
    \item Verify bibliography file syntax
    \item Ensure all required packages are installed
\end{itemize}

\section{Best Practices}
\label{sect:best-practices}

\subsection{File Organization}
\label{sect:file-organization}

Maintain a clear directory structure:

\begin{lstlisting}[basicstyle=\ttfamily\footnotesize]
thesis/
|-- thesis.tex              # Main document
|-- preamble.tex            # Configuration
|-- 0_frontmatter/          # Title, abstract, etc.
|-- 1_introduction/         # Chapter directories
|-- 2_methodology/
|-- bib/                    # Bibliography files
|-- figures/                # Figure files
+-- Latex/                  # Class and style files
\end{lstlisting}

\subsection{Version Control}
\label{sect:version-control}

Use version control (git) to track changes. The template includes appropriate \texttt{.gitignore} patterns for LaTeX auxiliary files.

\subsection{Backup Strategy}
\label{sect:backup-strategy}

Regularly backup your work, especially:
\begin{itemize}
    \item Source \texttt{.tex} files
    \item Bibliography \texttt{.bib} files  
    \item Figure source files
    \item Custom macros and style modifications
\end{itemize}

This template provides a robust foundation for academic thesis writing while maintaining flexibility for customization and extension.
\chapter{Macros}
\label{chap:Macros}
\newrefsegment
\glsresetall
\section{Glossary and Acronym Reference}
\label{sect:glossary-reference}

This section demonstrates the glossary system by showing each term used twice - first use (with full definition) and subsequent use (abbreviated form). The glossary system allows for consistent terminology usage throughout the document, and all terms used here will appear in the final glossary at the end of the thesis.

\subsection{Glossary Terms in Use}
\label{sect:glossary-terms}

The following table demonstrates all glossary entries by using them naturally. Each term is shown twice to illustrate the difference between first use and subsequent use formatting, plus the convenience command when available:

\begin{longtable}{p{0.3\textwidth}p{0.3\textwidth}p{0.3\textwidth}}
\caption{Glossary terms showing first use, subsequent use, and convenience commands} \\
\toprule
\textbf{First Use (Full Definition)} & \textbf{Subsequent Use (Abbreviated)} & \textbf{Convenience Command} \\
\midrule
\endfirsthead

\caption[]{Glossary terms (continued)} \\
\toprule
\textbf{First Use (Full Definition)} & \textbf{Subsequent Use (Abbreviated)} & \textbf{Convenience Command} \\
\midrule
\endhead

\midrule
\multicolumn{3}{r}{\textit{Continued on next page}} \\
\endfoot

\bottomrule
\endlastfoot

\gls{adu} & \gls{adu} & \texttt{\textbackslash ADU} produces \ADU \\
\gls{ao} & \gls{ao} & \texttt{\textbackslash AO} produces \AO \\
\gls{ar} & \gls{ar} & \texttt{\textbackslash AR} produces \AR \\
\gls{awg} & \gls{awg} & \texttt{\textbackslash awg} produces \awg \\
\gls{cots} & \gls{cots} & \texttt{\textbackslash COTS} produces \COTS \\
\gls{dft} & \gls{dft} & \texttt{\textbackslash DFT} produces \DFT \\
\gls{efl} & \gls{efl} & \texttt{\textbackslash efl} produces \efl \\
\gls{fft} & \gls{fft} & \texttt{\textbackslash fft} produces \fft \\
\gls{fp} & \gls{fp} & \texttt{\textbackslash FabryPerot} produces \FabryPerot \\
\gls{fsr} & \gls{fsr} & \texttt{\textbackslash fsr} produces \fsr \\
\gls{ft} & \gls{ft} & \texttt{\textbackslash ft} produces \ft \\
\gls{fts} & \gls{fts} & \texttt{\textbackslash fts} produces \fts \\
\gls{fwhm} & \gls{fwhm} & \texttt{\textbackslash fwhm} produces \fwhm \\
\gls{ifu} & \gls{ifu} & \texttt{\textbackslash ifu} produces \ifu \\
\gls{ips} & \gls{ips} & \texttt{\textbackslash ips} produces \ips \\
\gls{ir} & \gls{ir} & \texttt{\textbackslash ir} produces \ir \\
\gls{leo} & \gls{leo} & \texttt{\textbackslash leo} produces \leo \\
\gls{lsf} & \gls{lsf} & \texttt{\textbackslash lsf} produces \lsf \\
\gls{mcf} & \gls{mcf} & \texttt{\textbackslash mcf} produces \mcf \\
\gls{mfd} & \gls{mfd} & \texttt{\textbackslash mfd} produces \mfd \\
\gls{sm} & \gls{sm} & \texttt{SM} produces \sm \\
\gls{mm} & \gls{mm} & \texttt{\textbackslash MM} produces \MM \\
\gls{mmf} & \gls{mmf} & \texttt{\textbackslash MMF} produces \MMF \\
\gls{mos} & \gls{mos} & \texttt{\textbackslash mos} produces \mos \\
\gls{na} & \gls{na} & \texttt{\textbackslash NA} produces \NA \\
\gls{nir} & \gls{nir} & \texttt{\textbackslash nir} produces \nir \\
\gls{pcb} & \gls{pcb} & \texttt{\textbackslash pcb} produces \pcb \\
\gls{pcf} & \gls{pcf} & \texttt{\textbackslash pcf} produces \pcf \\
\gls{pl} & \gls{pl} & \texttt{\textbackslash lantern} produces \lantern \\
\gls{pop} & \gls{pop} & \texttt{\textbackslash pop} produces \pop \\
\gls{psf} & \gls{psf} & \texttt{\textbackslash psf} produces \psf \\
\gls{ptv} & \gls{ptv} & \texttt{\textbackslash ptv} produces \ptv \\
\gls{rms} & \gls{rms} & \texttt{\textbackslash rms} produces \rms \\
\gls{sm} & \gls{sm} & \texttt{\textbackslash sm} produces \sm \\
\gls{smf} & \gls{smf} & \texttt{\textbackslash smf} produces \smf \\
\gls{snr} & \gls{snr} & \texttt{\textbackslash snr} produces \snr \\
\gls{swir} & \gls{swir} & \texttt{\textbackslash swir} produces \swir \\
\gls{tec} & \gls{tec} & \texttt{\textbackslash tec} produces \tec \\
\gls{thar} & \gls{thar} & \texttt{\textbackslash thar} produces \thar \\
\gls{uli} & \gls{uli} & \texttt{\textbackslash uli} produces \uli \\
\gls{uv} & \gls{uv} & \texttt{\textbackslash uv} produces \uv \\
\gls{vph} & \gls{vph} & \texttt{\textbackslash vph} produces \vph \\
\gls{etendue} & \gls{etendue} & \texttt{\textbackslash etendue} produces \etendue \\
\gls{f-ratio} & \gls{f-ratio} & \texttt{\textbackslash fnum} produces \fnum \\
\gls{resolvingpower} & \gls{resolvingpower} & \texttt{\textbackslash R} produces \R \\
\gls{seeing} & \gls{seeing} & \texttt{\textbackslash seeing} produces \seeing \\
\gls{spectralresolution} & \gls{spectralresolution} & \texttt{\textbackslash dlam} produces \dlam \\
\gls{throughput} & \gls{throughput} & \texttt{\textbackslash throughput} produces \throughput \\
\end{longtable}

\subsection{Using Glossary Commands}
\label{sect:glossary-commands}

The template provides several commands for using glossary entries consistently:

\begin{description}
    \item[\texttt{\textbackslash gls\{key\}}] First use shows full definition, subsequent uses show abbreviated form
    \item[\texttt{\textbackslash glspl\{key\}}] Plural form of glossary entry (e.g., \glspl{smf})
    \item[\texttt{\textbackslash glsfirst\{key\}}] Always shows the first use form
    \item[\texttt{\textbackslash glstext\{key\}}] Shows only the short form/acronym
    \item[\texttt{\textbackslash glssymbol\{key\}}] Shows the symbol (e.g., \glssymbol{etendue})
    \item[\texttt{\textbackslash glsreset\{key\}}] Resets entry so next use shows full form
    \item[\texttt{\textbackslash glsresetall}] Resets all entries (used at chapter beginnings)
\end{description}

Many entries also have convenience commands defined, such as \texttt{\textbackslash smf} which expands to \smf, and \texttt{\textbackslash fwhm} which expands to \fwhm. These convenience commands provide a shorter way to reference common terms.

\subsubsection{Plural Forms}
\label{sect:plural-forms}

Many glossary entries support plural forms using \texttt{\textbackslash glspl\{key\}}. Some entries also have dedicated plural convenience commands:

\begin{itemize}
    \item \texttt{\textbackslash glspl\{smf\}} produces: \glspl{smf}
    \item \texttt{\textbackslash smfs} convenience command produces: \smfs
    \item \texttt{\textbackslash glspl\{mmf\}} produces: \glspl{mmf}  
    \item \texttt{\textbackslash MMFs} convenience command produces: \MMFs
    \item \texttt{\textbackslash glspl\{psf\}} produces: \glspl{psf}
    \item \texttt{\textbackslash psfs} convenience command produces: \psfs
    \item \texttt{\textbackslash glspl\{pl\}} produces: \glspl{pl}
    \item \texttt{\textbackslash lanterns} convenience command produces: \lanterns
\end{itemize}

The plural forms automatically handle the appropriate grammatical form for each term, including both the abbreviated and full forms on first use.

\section{Demo Text}
Citations for demo at end: \autocite{yourname2024spie1,yourname2023spie2,yourname2022spie4}

\chapter{Figures}
\label{chap:DesignPrinciples}
\newrefsegment
\glsresetall

\section{Figure examples and macros}
\label{sect:figexamples}

This chapter demonstrates the use of various figure referencing macros and figure insertion commands provided by the thesis template. The template includes convenient macros that enforce consistent labeling and referencing practices throughout your thesis.

The experimental techniques discussed here build upon recent advances in data analysis for high-energy physics experiments \autocite{yourname2024photwest} and modern condensed matter physics methodologies \autocite{yourname2023labchar}. These approaches have become standard practice in contemporary experimental physics research.

\subsection{Cross-Referencing Macros}
\label{sect:crossref}

The template provides several referencing macros defined in MacroFile1.tex that automatically format references with proper prefixes. These referencing conventions are consistent with those used in the experimental validation of novel computational methods \autocite{yourname2023mnras} and numerical analysis studies \autocite{rodriguez2023opex}:

\begin{itemize}
    \item \texttt{\textbackslash fig\{label\}} produces short figure references like \fig{example-image-a}
    \item \texttt{\textbackslash Fig\{label\}} produces long figure references like \Fig{example-image-b}
    \item \texttt{\textbackslash eqn\{label\}} produces short equation references like \eqn{sampleequation}
    \item \texttt{\textbackslash Eqn\{label\}} produces long equation references like \Eqn{sampleequation}
    \item \texttt{\textbackslash sect\{label\}} produces section references like \sect{figexamples}
    \item \texttt{\textbackslash chap\{label\}} produces chapter references like \chap{intro}
    \item \texttt{\textbackslash tabl\{label\}} produces table references like \tabl{sampletable}
\end{itemize}

\subsection{Sample Equation}
Here is a sample equation to demonstrate equation referencing:

\begin{equation}
    E = mc^2
    \label{eqn:sampleequation}
\end{equation}

As shown in \Eqn{sampleequation}, Einstein's mass-energy equivalence is one of the most famous equations in physics. You can also reference it as \eqn{sampleequation} for a shorter form.

\subsection{Figure Insertion Macros}
\label{sect:figmacros}

The template provides several convenient macros for inserting figures consistently. The following examples demonstrate the different figure macros available:

\subsubsection{Dual Figure Layout}
The \texttt{\textbackslash figuremacroSUB} macro creates side-by-side figures as shown below:

\figuremacroSUB{Comparison of example images}{example-image-a}{This shows the first example image demonstrating the left panel of a dual figure layout}{0.4}{example-image-b}{This shows the second example image demonstrating the right panel of a dual figure layout}{0.4}

Notice how \Fig{Comparison of example images} demonstrates the dual figure layout. You can reference individual components using \fig{example-image-a} and \fig{example-image-b}.

\subsubsection{Single Figure with Width Control}
The \texttt{\textbackslash figuremacroW} macro provides width control for single figures:

\figuremacroW{example-image-c}{Full-width example}{This figure demonstrates a full-width image using the figuremacroW command with width parameter set to 1.0}{1}

The full-width layout shown in \fig{example-image-c} is useful for wide diagrams or detailed images that need maximum space.

\subsubsection{Figure with Position Control}
The \texttt{\textbackslash figuremacroWp} macro adds position control parameters:

\figuremacroWp{example-image-a}{Positioned example}{This figure demonstrates position control using the figuremacroWp macro with width 0.8 and position 'h' (here)}{0.8}{h}

\subsection{Sample Table}
\label{sect:sampletable}

Here's a sample table to demonstrate table referencing:

\begin{table}[h]
\centering
\begin{tabular}{|l|c|r|}
\hline
\textbf{Macro} & \textbf{Output Format} & \textbf{Example} \\
\hline
\texttt{\textbackslash fig\{label\}} & Fig.~\textbackslash ref\{fig:label\} & Fig.~1.1 \\
\texttt{\textbackslash Fig\{label\}} & Figure~\textbackslash ref\{fig:label\} & Figure~1.1 \\
\texttt{\textbackslash eqn\{label\}} & Eqn.~\textbackslash ref\{eqn:label\} & Eqn.~1.1 \\
\texttt{\textbackslash sect\{label\}} & Sec.~\textbackslash ref\{sect:label\} & Sec.~1.1 \\
\hline
\end{tabular}
\caption{Summary of referencing macros and their output formats}
\label{tabl:sampletable}
\end{table}

As detailed in \tabl{sampletable}, each macro provides consistent formatting for different reference types.

\begin{figure}[p]
\centering
\def\svgwidth{0.8\columnwidth}
\includesvg{2/figures/1x7}
	\caption[TOC title]{Caption.}
	\label{fig:MCFgeometry}
\end{figure}

\begin{figure}[p]
	\centering
	\includegraphics[width=0.4\textwidth]{example-grid-100x100pt}
	\caption[TOC Title]{Long Caption}
	\label{fig:simulated tiger}
\end{figure}



\todofigure{Figure Placeholder}








% --------------------------------------------------------------
%:                  BACK MATTER: appendices, refs,..
% --------------------------------------------------------------

% the back matter: appendix and references close the thesis


%: ----------------------- bibliography ------------------------

% The section below defines how references are listed and formatted
% The default below is 2 columns, small font, complete author names.
% Entries are also linked back to the page number in the text and to external URL if provided in the BibTex file.

%\nocite{*}

\renewcommand*{\bibfont}{\scriptsize}
\renewcommand{\bibname}{References}
\printbibheading[heading=bibintoc]

%\begin{multicols}{2}
%\begin{footnotesize}
%
%\printbibliography[segment=0,notcategory=fullcited]
\bibbysegment[heading=subbibliography]
%
%\end{footnotesize}
%\end{multicols}

%%: ----------------------- glossary ------------------------
%
%% Tie in external source file for definitions: /0_frontmatter/glossary.tex
%% Glossary entries can also be defined in the main text. See glossary.tex

%\glsaddall
\printglossary[style=mcoltreegroup, title=Acronyms and Definitions]



%: ----------------------- appendix -------------------------

% according to Dresden med fac summary has to be at the end

%\newgeometry{textwidth=18.5cm,textheight=20.45cm}
\appendix
\vspace{-1.0cm}

\chapter{MISPRINT Class}\label{chap:MISPRINTcode}
%\lstinputlisting[caption=MISPRINT primary class,
%			  basicstyle=\tiny,
%			  label{apdx:code:MISPRINTcode}]{/Users/chrisbetters/Sydney Uni Dropbox/Chris Betters/github/common/+chrislib/@misprint/misprint.m}
			  
%'/Users/chrisbetters/Sydney Uni Dropbox/Chris Betters/github/common/+chrislib/@misprint'			  
%\input{/Users/chrisbetters/github/phd/+chrislib/@misprint/html/misprint.tex}
		
			  
\chapter{ZEMAX Macro to setup \echelle wavelength and orders}
	\lstdefinelanguage{zpl}{morekeywords={ABSO, ACOS, APMN, APMX, APXD, APYD, APTP, ASIN, ASPR, ATAN, ATYP, AVAL, CONF, CONI, COSI, CURV, EDGE, EOFF, ETIM, EXPE, EXPT, FICL, FLDX, FLDY, FTYP, FVAN, FVCX, FVCY, FVDX, FVDY, FWGT, GABB, GETT, GIND, GLCA, GLCB, GLCC, GLCM, GLCX, GLCY, GLCZ, GNUM, GPAR, GRIN, IMAE, INDX, INTE, ISMS, LOGE, LOGT, MAGN, MAXF, MAXG, MCON, MCOP, MFCN, NCON, NFLD, NOBJ, NPAR, NPOS, NPRO, NSDC, NSDD, NSUR, NWAV, OCOD, ONUM, OPDC, OPER, OPEV, OPTH, PARM, PAR, PMOD, POWR, PVHX, PVHY, PVPX, PVPY, PWAV, RADI, RAGX, RAGY, RAGZ, RAND, RANX, RANY, RANZ, RAYE, RAYL, RAYM, RAYN, RAYO, RAYT, RAYV, RAYX, RAYY, RAYZ, RELI, SAGG, SCOD, SCOM, SDIA, SIGN, SINE, SLEN, SPRO, SQRT, STDD, STYP, SVAL, TANG, TMAS, THIC, UNIT, VEC1, VEC2, VEC3, VEC4, VERS, WAVL, WWGT, XMIN, XMAX, YMIN, YMAX, APMN, APMX, APXD, APYD, APTP, ATYP, AVAL, BEEP, COAT, CLOSE, CLOSEWINDOW, COLOUR, COMMENT, CONI, COPYFILE, CURV, DELETE, DELETECONFIG, DELETEFILE, DELETEMCO, DELETEMFO, DELETEOBJECT, EDVA, END, EXPORTBMP, EXPORTCAD, EXPORTJPG, EXPORTWMF, FINDFILE, FLDX, FLYD, FWGT, FVDX, FVDY, FOR, NEXT, FORMAT, FTYP, GCRS, GDATE, GETEXTRADATA, GETGLASSDATA, GETLSF, GETMTF, GETPSF, GETSYSTEMDATA, GETTEXTFILE, GETVARDATA, GETZERNIKE, GLAS, GLENSNAME, GOSUB, SUB, RETURN, END, GOTO, GRAPHICS, GTEXT, GTEXTCENT, GTITLE, HAMMER, IF, THEN, ELSE, ENDIF, IMA, IMASHOW, IMASUM, IMPORT_ED, INPUT, INSERT, INSERTCONFIG, INSERTMCO, INSERTMFO, INSERTOBJECT, LABEL, LINE, LOADCATALOG, LOADLENS, LOADMERIT, LOCKWINDOW, NEXT, NSTR, NUMFIELD, NUMWAVE, OPEN, OPTIMIZE, OPTRETURN, OUTPUT, PARM, PAR, PARAXIAL, PAUSE, PIXLE, POLDEFINE, POLTRACE, POP, PRINT, PRINTFILE, PRINTWINDOW, PWAV, QUICKFOCUS, RADI, RANDOMIZE, RAYTRACE, RAYTRACEX, READ, READNEXT, READSTRING, RELOADOBJECTS, REM, REMOVEVARIABLES, RENAMEFILE, RETURN, REWIND, SAVELENS, SAVEMERIT, SAVEWINDOW, SCATTER, SDIA, SETAIM, SETAIMDATA, SETAPODIZATION, SETCONFIG, SETDETECTOR, SETMCOPERAND, SETNSCPARAMETER, SETNSCPOSTION, SETNSCPROPERTY, SETOPERAND, SETSTDD, SETSURFACEPROPERTY, SURP, SETSYSTEMPROPERTY, SYSP, SETTEXTSIZE, SETTILE, SETUNITS, SETVAR, SETVECSIZE, SETVIG, SHOWFILE, SOLVETYPE, STOPSURF, SUB, SURFTYPE, TELECENTRIC, TESTPLATEFIT, THIC, TIMER, UNLOCKWINDOW, UPDATE, VEC1, VEC2, VEC3, VEC4, WAVL, WWGT, XDIFFIA, ZBFCLR, ZBFMULT, ZBFPROPERTIES, ZBFRESAMPLE, ZBFSHOW, ZBFSUM, ZBFTILT,CALLSETDBL,CALLMACRO,CALD},
morecomment=[l]{\#},
morecomment=[l]{!},
morestring=[b]"}

%\lstinputlisting[caption={ZEMAX macro to fill appropriate wavelengths for \echelle orders in separate configurations. \label{apdx:code:ZEMAX_echelle}},
%			  basicstyle=\tiny,
%			    language=zpl,
%			  breaklines=true]{/Users/chrisbetters/Dropbox/ZEMAX/Macros/echelle_setup.zpl}
%
%\input{2/radiation_enviroment}

\chapter{Code Examples}\label{chap:code_examples}

\section{Python Class Example}
\lstinputlisting[caption={Example Python class for optical system analysis},
			  language=Python,
			  basicstyle=\tiny,
			  label={apdx:code:python_example}]{9_backmatter/python_example.py}

\section{MATLAB Class Example}
\lstinputlisting[caption={Example MATLAB class for wavefront analysis},
			  basicstyle=\tiny,
			  label={apdx:code:matlab_example}]{9_backmatter/matlab_example.m}

\section{ZEMAX ZPL Macro Example}
\lstinputlisting[caption={Example ZEMAX ZPL macro for automated analysis},
			  language=zpl,
			  basicstyle=\tiny,
			  breaklines=true,
			  label={apdx:code:zemax_example}]{9_backmatter/zemax_example.zpl}
\chapter{Included Papers}
\insertpaper{{Paper Title}}{pdf:paper1}{9_backmatter/papers/style-and-template-for-preprints-arxiv-bio-arxiv.pdf}


\end{document}
