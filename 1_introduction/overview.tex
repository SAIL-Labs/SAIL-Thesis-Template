\chapter{Template Overview and Usage Guide}
\label{chap:overview}
\newrefsegment
\glsresetall

This chapter provides a comprehensive guide for using this LaTeX thesis template. The template is designed for University of Sydney PhD theses and includes advanced features for figure management, bibliography handling, cross-referencing, and document customization.

\section{Getting Started}
\label{sect:getting-started}

To begin using this template, you need to customize several key files with your specific thesis information. This section outlines the essential modifications required.

\subsection{Document Metadata Configuration}
\label{sect:metadata-config}

The primary customization occurs in \texttt{preamble.tex}, where you define your thesis metadata. Update the following variables:

\begin{lstlisting}[language=TeX,caption={Document metadata in preamble.tex}]
% Document setup using dynamic variables
\title{Your amazing thesis title.}
\author{First Lastname}
\def\thesisauthoremail{First.Lastname@sydney.edu.au}
\def\thesiskeywords{Science; PhD Thesis}

\collegeordept{Faculty of Science}
\university{The University of Sydney}
\degree{Doctor of Philosophy}
\degreedate{\monthyeardate\today}
\degreeyear{\the\year}
\end{lstlisting}

These variables automatically populate the title page, PDF metadata, and hyperlink information throughout your document.

\subsection{Draft Mode Configuration}
\label{sect:draft-mode}

The template supports draft mode for faster compilation during development. In \texttt{thesis.tex}, you can toggle draft mode:

\begin{lstlisting}[language=TeX,caption={Draft mode toggle in thesis.tex}]
% Toggle draft mode - figures show as placeholder boxes
\thesisisdrafttrue  % Enable draft mode
% \thesisisdraftfalse % Disable draft mode for final version
\end{lstlisting}

When draft mode is enabled, all figures display as placeholder boxes, significantly reducing compilation time.

\section{Bibliography Management}
\label{sect:bibliography}

The template uses the biblatex package with biber backend for sophisticated bibliography management.

\subsection{Bibliography Files}
\label{sect:bib-files}

Bibliography sources are configured in \texttt{preamble.tex}:

\begin{lstlisting}[language=TeX,caption={Bibliography configuration}]
\addbibresource{bib/thesisRefs.bib}
\addbibresource{bib/fake_publications.bib}
\end{lstlisting}

Add your own \texttt{.bib} files to the \texttt{bib/} directory and include them here.

\subsection{Citation Commands}
\label{sect:citations}

The template uses biblatex citation commands. Key commands include:

\begin{itemize}
    \item \texttt{\textbackslash autocite\{key\}} - Automatic citation format: \autocite{yourname2024nature}
    \item \texttt{\textbackslash textcite\{key\}} - In-text citation: \textcite{yourname2024nature}
    \item \texttt{\textbackslash parencite\{key\}} - Parenthetical citation: \parencite{yourname2024nature}
    \item \texttt{\textbackslash fullcite\{key\}} - Full citation in text
    \item \texttt{\textbackslash fullciteandexclude\{key\}} - Full citation excluded from bibliography
\end{itemize}

\subsection{Compilation Sequence}
\label{sect:compilation}

For proper bibliography processing, use this compilation sequence:

\begin{lstlisting}[language=bash,caption={Complete compilation sequence}]
pdflatex thesis.tex
biber thesis
pdflatex thesis.tex
pdflatex thesis.tex  # Second run for cross-references
\end{lstlisting}

\section{Figure Management}
\label{sect:figure-management}

The template provides powerful figure insertion macros defined in \texttt{Latex/Macros/MacroFile1.tex}.

\subsection{Figure Insertion Macros}
\label{sect:figure-macros}

Key figure macros include:

\begin{description}
    \item[\texttt{\textbackslash figuremacroW\{filename\}\{label\}\{caption\}\{width\}}] Single figure with width control
    \item[\texttt{\textbackslash figuremacroWp\{filename\}\{label\}\{caption\}\{width\}\{position\}}] Single figure with position control
    \item[\texttt{\textbackslash figuremacroSUB\{label\}\{file1\}\{caption1\}\{width1\}\{file2\}\{caption2\}\{width2\}}] Side-by-side figures
    \item[\texttt{\textbackslash figuremacroSUBp\{...\}\{position\}}] Side-by-side with position control
\end{description}

These macros automatically handle figure placement, labeling, and formatting consistency.

\subsection{Cross-Referencing System}
\label{sect:cross-ref-system}

The template provides consistent cross-referencing macros:

\begin{description}
    \item[\texttt{\textbackslash fig\{label\}}] Short figure reference (Fig.~X.X)
    \item[\texttt{\textbackslash Fig\{label\}}] Long figure reference (Figure~X.X)
    \item[\texttt{\textbackslash eqn\{label\}}] Short equation reference (Eqn.~X.X)
    \item[\texttt{\textbackslash Eqn\{label\}}] Long equation reference (Equation~X.X)
    \item[\texttt{\textbackslash sect\{label\}}] Section reference (Sec.~X.X)
    \item[\texttt{\textbackslash chap\{label\}}] Chapter reference (Chap.~X)
    \item[\texttt{\textbackslash tabl\{label\}}] Table reference (Table~X.X)
\end{description}

\section{Chapter Organization}
\label{sect:chapter-organization}

\subsection{Chapter Structure}
\label{sect:chapter-structure}

Each chapter should follow this structure:

\begin{lstlisting}[language=TeX,caption={Standard chapter template}]
\chapter{Chapter Title}
\label{chap:chapter-label}
\newrefsegment  % Start new reference segment
\glsresetall    % Reset glossary entries

% Chapter content here
\end{lstlisting}

The \texttt{\textbackslash newrefsegment} command enables per-chapter bibliography sections if needed, while \texttt{\textbackslash glsresetall} resets glossary entry formatting.

\subsection{Adding New Chapters}
\label{sect:adding-chapters}

To add a new chapter:

\begin{enumerate}
    \item Create a new \texttt{.tex} file in an appropriate directory
    \item Add the chapter structure shown above
    \item Include the file in \texttt{thesis.tex} using \texttt{\textbackslash include\{path/to/chapter\}} (without .tex extension)
    \item Update the table of contents if necessary
\end{enumerate}

\subsection{Selective Compilation with \texttt{\textbackslash includeonly}}
\label{sect:includeonly}

When working on a large thesis, you may want to compile only specific chapters to save time. The template supports this through LaTeX's \texttt{\textbackslash includeonly} command.

\subsubsection{Using \texttt{\textbackslash includeonly} in Overleaf}

In the main \texttt{thesis.tex} file, you'll find commented examples:

\begin{lstlisting}[language=TeX,caption={includeonly examples in thesis.tex}]
%\includeonly{0_frontmatter/abstract,0_frontmatter/list_of_publications}
% \includeonly{1_introduction/overview}
% \includeonly{2/macros}
% \includeonly{3/figures}
\end{lstlisting}

To compile only specific chapters:

\begin{enumerate}
    \item \textbf{Uncomment} the \texttt{\textbackslash includeonly} line for your desired chapters
    \item \textbf{Ensure only one} \texttt{\textbackslash includeonly} command is active at a time
    \item \textbf{Recompile} in Overleaf - only the specified chapters will be processed
    \item \textbf{Comment out} the \texttt{\textbackslash includeonly} line when you want to compile the full document
\end{enumerate}

\subsubsection{Benefits for Overleaf Users}

\begin{itemize}
    \item \textbf{Faster compilation}: Large theses with many figures compile much faster when working on individual chapters
    \item \textbf{Reduced timeout risk}: Overleaf has compilation time limits; selective compilation helps stay within them
    \item \textbf{Preserved cross-references}: LaTeX maintains references to other chapters even when they're not compiled
    \item \textbf{Efficient editing}: Focus on specific sections without waiting for the entire document to compile
\end{itemize}

\subsubsection{Important Notes}

\begin{itemize}
    \item Use \texttt{\textbackslash include\{path/filename\}} not \texttt{\textbackslash input\{path/filename\}} for chapters you want to selectively compile
    \item The file path in \texttt{\textbackslash includeonly} must exactly match the path in \texttt{\textbackslash include}
    \item Cross-references to excluded chapters will show as ``??'' until you compile the full document
    \item Always do a final compilation with \texttt{\textbackslash includeonly} commented out before submission
\end{itemize}

\subsubsection{Example Workflow in Overleaf}

\begin{enumerate}
    \item Working on methodology chapter: Uncomment \texttt{\textbackslash includeonly\{2/macros\}}
    \item Make edits to your methodology content
    \item Compile quickly to see changes (faster than full document)
    \item When ready to see full document with all references: Comment out \texttt{\textbackslash includeonly}
    \item Final compile processes entire thesis with all cross-references resolved
\end{enumerate}

\section{Publications and Awards}
\label{sect:publications-awards}

\subsection{List of Publications}
\label{sect:list-publications}

The template includes \texttt{0\_frontmatter/list\_of\_publications.tex} for showcasing your academic contributions. This file uses:

\begin{lstlisting}[language=TeX,caption={Publication listing format}]
\section*{Journal Papers}
\begin{itemize}
    \item \fullciteandexclude{paper-key}
    \\ Contents included throughout \chap{chapter-label} and \sect{section-label}.
\end{itemize}
\end{lstlisting}

The \texttt{\textbackslash fullciteandexclude\{key\}} command displays the full citation but excludes it from the main bibliography to avoid duplication.

\subsection{Connecting Publications to Content}
\label{sect:connecting-publications}

For each publication, specify which chapters and sections incorporate that work. This demonstrates the connection between your published research and thesis content.

\section{Advanced Features}
\label{sect:advanced-features}

\subsection{Glossaries and Acronyms}
\label{sect:glossaries}

The template supports glossaries through the \texttt{glossaries} package. Define terms in glossary files and compile with:

\begin{lstlisting}[language=bash]
makeglossaries thesis
\end{lstlisting}

\subsection{Code Listings}
\label{sect:code-listings}

The template includes support for code listings with syntax highlighting:

\begin{lstlisting}[language=TeX,caption={Code listing example}]
\begin{lstlisting}[language=Python,caption={Python example}]
def hello_world():
    print("Hello, World!")
\end{lstlisting}

\subsection{Mathematical Content}
\label{sect:math-content}

The template loads \texttt{amsmath} and related packages for comprehensive mathematical typesetting. Use standard LaTeX math environments:

\begin{equation}
    \nabla \times \mathbf{E} = -\frac{\partial \mathbf{B}}{\partial t}
    \label{eqn:maxwell-faraday}
\end{equation}

Reference equations using \texttt{\textbackslash eqn\{maxwell-faraday\}} or \texttt{\textbackslash Eqn\{maxwell-faraday\}}.

\section{Customization Options}
\label{sect:customization}

\subsection{Color Scheme}
\label{sect:color-scheme}

University colors are defined in \texttt{preamble.tex}:

\begin{lstlisting}[language=TeX,caption={Color definitions}]
\definecolor{SchoolColor}{RGB}{230, 70, 38} % Masterbrand red
\definecolor{chaptergrey}{RGB}{230, 70, 38}
\definecolor{midgrey}{RGB}{66, 66, 66}
\end{lstlisting}

Modify these values to match your requirements while maintaining university branding guidelines.

\subsection{Font Configuration}
\label{sect:font-config}

The template uses Computer Modern fonts by default. Font modifications should be made carefully to maintain readability and university requirements.

\subsection{Page Layout}
\label{sect:page-layout}

Page layout parameters are defined in \texttt{Latex/Classes/PhDthesisPSnPDF.cls}. Modify margins and spacing only if required by university guidelines.

\section{Troubleshooting}
\label{sect:troubleshooting}

\subsection{Common Issues}
\label{sect:common-issues}

\begin{description}
    \item[Bibliography not appearing] Ensure you run \texttt{biber} after \texttt{pdflatex}
    \item[Cross-references showing as ??] Run \texttt{pdflatex} multiple times for cross-reference resolution
    \item[Figures not displaying] Check file paths and ensure draft mode is disabled
    \item[Glossary not generating] Run \texttt{makeglossaries} and recompile
\end{description}

\subsection{Compilation Errors}
\label{sect:compilation-errors}

For complex documents, compilation errors may occur. Common solutions:

\begin{itemize}
    \item Clear auxiliary files (\texttt{.aux}, \texttt{.bbl}, \texttt{.blg}, etc.) and recompile
    \item Check for undefined references or missing files
    \item Verify bibliography file syntax
    \item Ensure all required packages are installed
\end{itemize}

\section{Best Practices}
\label{sect:best-practices}

\subsection{File Organization}
\label{sect:file-organization}

Maintain a clear directory structure:

\begin{lstlisting}[basicstyle=\ttfamily\footnotesize]
thesis/
|-- thesis.tex              # Main document
|-- preamble.tex            # Configuration
|-- 0_frontmatter/          # Title, abstract, etc.
|-- 1_introduction/         # Chapter directories
|-- 2_methodology/
|-- bib/                    # Bibliography files
|-- figures/                # Figure files
+-- Latex/                  # Class and style files
\end{lstlisting}

\subsection{Version Control}
\label{sect:version-control}

Use version control (git) to track changes. The template includes appropriate \texttt{.gitignore} patterns for LaTeX auxiliary files.

\subsection{Backup Strategy}
\label{sect:backup-strategy}

Regularly backup your work, especially:
\begin{itemize}
    \item Source \texttt{.tex} files
    \item Bibliography \texttt{.bib} files  
    \item Figure source files
    \item Custom macros and style modifications
\end{itemize}

This template provides a robust foundation for academic thesis writing while maintaining flexibility for customization and extension.